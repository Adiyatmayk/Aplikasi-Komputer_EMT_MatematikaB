% Options for packages loaded elsewhere
\PassOptionsToPackage{unicode}{hyperref}
\PassOptionsToPackage{hyphens}{url}
\documentclass[
]{book}
\usepackage{xcolor}
\usepackage{amsmath,amssymb}
\setcounter{secnumdepth}{-\maxdimen} % remove section numbering
\usepackage{iftex}
\ifPDFTeX
  \usepackage[T1]{fontenc}
  \usepackage[utf8]{inputenc}
  \usepackage{textcomp} % provide euro and other symbols
\else % if luatex or xetex
  \usepackage{unicode-math} % this also loads fontspec
  \defaultfontfeatures{Scale=MatchLowercase}
  \defaultfontfeatures[\rmfamily]{Ligatures=TeX,Scale=1}
\fi
\usepackage{lmodern}
\ifPDFTeX\else
  % xetex/luatex font selection
\fi
% Use upquote if available, for straight quotes in verbatim environments
\IfFileExists{upquote.sty}{\usepackage{upquote}}{}
\IfFileExists{microtype.sty}{% use microtype if available
  \usepackage[]{microtype}
  \UseMicrotypeSet[protrusion]{basicmath} % disable protrusion for tt fonts
}{}
\makeatletter
\@ifundefined{KOMAClassName}{% if non-KOMA class
  \IfFileExists{parskip.sty}{%
    \usepackage{parskip}
  }{% else
    \setlength{\parindent}{0pt}
    \setlength{\parskip}{6pt plus 2pt minus 1pt}}
}{% if KOMA class
  \KOMAoptions{parskip=half}}
\makeatother
\usepackage{graphicx}
\makeatletter
\newsavebox\pandoc@box
\newcommand*\pandocbounded[1]{% scales image to fit in text height/width
  \sbox\pandoc@box{#1}%
  \Gscale@div\@tempa{\textheight}{\dimexpr\ht\pandoc@box+\dp\pandoc@box\relax}%
  \Gscale@div\@tempb{\linewidth}{\wd\pandoc@box}%
  \ifdim\@tempb\p@<\@tempa\p@\let\@tempa\@tempb\fi% select the smaller of both
  \ifdim\@tempa\p@<\p@\scalebox{\@tempa}{\usebox\pandoc@box}%
  \else\usebox{\pandoc@box}%
  \fi%
}
% Set default figure placement to htbp
\def\fps@figure{htbp}
\makeatother
\setlength{\emergencystretch}{3em} % prevent overfull lines
\providecommand{\tightlist}{%
  \setlength{\itemsep}{0pt}\setlength{\parskip}{0pt}}
\usepackage{bookmark}
\IfFileExists{xurl.sty}{\usepackage{xurl}}{} % add URL line breaks if available
\urlstyle{same}
\hypersetup{
  hidelinks,
  pdfcreator={LaTeX via pandoc}}

\author{}
\date{}

\begin{document}
\frontmatter

\mainmatter
\chapter{Visualisasi dan Perhitungan Geometri dengan EMT}\label{visualisasi-dan-perhitungan-geometri-dengan-emt}

Euler menyediakan beberapa fungsi untuk melakukan visualisasi dan perhitungan geometri, baik secara numerik maupun analitik (seperti biasanya tentunya, menggunakan Maxima). Fungsi-fungsi untuk visualisasi dan perhitungan geometeri tersebut disimpan di dalam file program ``geometry.e'', sehingga file tersebut harus dipanggil sebelum menggunakan fungsi-fungsi atau perintah-perintah untuk geometri.

\textgreater load geometry

\begin{verbatim}
Numerical and symbolic geometry.
\end{verbatim}

\section{Fungsi-fungsi Geometri}\label{fungsi-fungsi-geometri}

Fungsi-fungsi untuk Menggambar Objek Geometri:

defaultd:=textheight()*1.5: nilai asli untuk parameter d\\
setPlotrange(x1,x2,y1,y2): menentukan rentang x dan y pada bidang koordinat\\
setPlotRange(r): pusat bidang koordinat (0,0) dan batas-batas sumbu-x dan y adalah -r sd r\\
plotPoint (P, ``P''): menggambar titik P dan diberi label ``P''\\
plotSegment (A,B, ``AB'', d): menggambar ruas garis AB, diberi label ``AB'' sejauh d\\
plotLine (g, ``g'', d): menggambar garis g diberi label ``g'' sejauh d\\
plotCircle (c,``c'',v,d): Menggambar lingkaran c dan diberi label ``c''\\
plotLabel (label, P, V, d): menuliskan label pada posisi P

Fungsi-fungsi Geometri Analitik (numerik maupun simbolik):

turn(v, phi): memutar vektor v sejauh phi\\
turnLeft(v): memutar vektor v ke kiri\\
turnRight(v): memutar vektor v ke kanan\\
normalize(v): normal vektor v\\
crossProduct(v, w): hasil kali silang vektorv dan w.\\
lineThrough(A, B): garis melalui A dan B, hasilnya {[}a,b,c{]} sdh. ax+by=c.\\
lineWithDirection(A,v): garis melalui A searah vektor v\\
getLineDirection(g): vektor arah (gradien) garis g\\
getNormal(g): vektor normal (tegak lurus) garis g\\
getPointOnLine(g): titik pada garis g\\
perpendicular(A, g): garis melalui A tegak lurus garis g\\
parallel (A, g): garis melalui A sejajar garis g\\
lineIntersection(g, h): titik potong garis g dan h\\
projectToLine(A, g): proyeksi titik A pada garis g\\
distance(A, B): jarak titik A dan B\\
distanceSquared(A, B): kuadrat jarak A dan B\\
quadrance(A, B): kuadrat jarak A dan B\\
areaTriangle(A, B, C): luas segitiga ABC\\
computeAngle(A, B, C): besar sudut \textless ABC\\
angleBisector(A, B, C): garis bagi sudut \textless ABC\\
circleWithCenter (A, r): lingkaran dengan pusat A dan jari-jari r\\
getCircleCenter(c): pusat lingkaran c\\
getCircleRadius(c): jari-jari lingkaran c\\
circleThrough(A,B,C): lingkaran melalui A, B, C\\
middlePerpendicular(A, B): titik tengah AB\\
lineCircleIntersections(g, c): titik potong garis g dan lingkran c\\
circleCircleIntersections (c1, c2): titik potong lingkaran c1 dan c2\\
planeThrough(A, B, C): bidang melalui titik A, B, C

Fungsi-fungsi Khusus Untuk Geometri Simbolik:

getLineEquation (g,x,y): persamaan garis g dinyatakan dalam x dan y\\
getHesseForm (g,x,y,A): bentuk Hesse garis g dinyatakan dalam x dan y dengan titik A pada

sisi positif (kanan/atas) garis

quad(A,B): kuadrat jarak AB

spread(a,b,c): Spread segitiga dengan panjang sisi-sisi a,b,c, yakni sin(alpha)\^{}2 dengan alpha sudut yang menghadap sisi a.

crosslaw(a,b,c,sa): persamaan 3 quads dan 1 spread pada segitiga dengan panjang sisi a, b, c.

triplespread(sa,sb,sc): persamaan 3 spread sa,sb,sc yang memebntuk suatu segitiga

doublespread(sa): Spread sudut rangkap Spread 2*phi, dengan sa=sin(phi)\^{}2 spread a.

\section{Contoh 1: Luas, Lingkaran Luar, Lingkaran Dalam Segitiga}\label{contoh-1-luas-lingkaran-luar-lingkaran-dalam-segitiga}

Untuk menggambar objek-objek geometri, langkah pertama adalah menentukan rentang sumbu-sumbu koordinat. Semua objek geometri akan digambar pada satu bidang koordinat, sampai didefinisikan bidang koordinat yang baru.

\textgreater setPlotRange(-0.5,2.5,-0.5,2.5); // mendefinisikan bidang koordinat baru

Sekarang tetapkan tiga poin dan plot mereka.

\textgreater A={[}1,0{]}; plotPoint(A,``A''); // definisi dan gambar tiga titik

\textgreater B={[}0,1{]}; plotPoint(B,``B'');

\textgreater C={[}2,2{]}; plotPoint(C,``C'');

Kemudian tiga segmen.

\textgreater plotSegment(A,B,``c''); // c=AB

\textgreater plotSegment(B,C,``a''); // a=BC

\textgreater plotSegment(A,C,``b''); // b=AC

Fungsi geometri meliputi fungsi untuk membuat garis dan lingkaran. Format garis adalah {[}a,b,c{]}, yang mewakili garis dengan persamaan ax+by=c.

\textgreater lineThrough(B,C) // garis yang melalui B dan C

\begin{verbatim}
[-1,  2,  2]
\end{verbatim}

Hitunglah garis tegak lurus yang melalui A pada BC.

\textgreater h=perpendicular(A,lineThrough(B,C)); // garis h tegak lurus BC melalui A

Dan persimpangannya dengan BC.

\textgreater D=lineIntersection(h,lineThrough(B,C)); // D adalah titik potong h dan BC

Plot itu.

\textgreater plotPoint(D,value=1); // koordinat D ditampilkan

\textgreater aspect(1); plotSegment(A,D): // tampilkan semua gambar hasil plot\ldots()

\begin{figure}
\centering
\pandocbounded{\includegraphics[keepaspectratio]{images/Adiyatma_23030630062__EMT4Geometry-001.png}}
\caption{images/Adiyatma\_23030630062\_\_EMT4Geometry-001.png}
\end{figure}

Hitung luas ABC:

\textgreater norm(A-D)*norm(B-C)/2 // AD=norm(A-D), BC=norm(B-C)

\begin{verbatim}
1.5
\end{verbatim}

Bandingkan dengan rumus determinan.

\textgreater areaTriangle(A,B,C) // hitung luas segitiga langusng dengan fungsi

\begin{verbatim}
1.5
\end{verbatim}

Cara lain menghitung luas segitigas ABC:

\textgreater distance(A,D)*distance(B,C)/2

\begin{verbatim}
1.5
\end{verbatim}

Sudut di C

\textgreater degprint(computeAngle(B,C,A))

\begin{verbatim}
36°52'11.63''
\end{verbatim}

Sekarang lingkaran luar segitiga.

\textgreater c=circleThrough(A,B,C); // lingkaran luar segitiga ABC

\textgreater R=getCircleRadius(c); // jari2 lingkaran luar

\textgreater O=getCircleCenter(c); // titik pusat lingkaran c

\textgreater plotPoint(O,``O''); // gambar titik ``O''

\textgreater plotCircle(c,``Lingkaran luar segitiga ABC''):

\begin{figure}
\centering
\pandocbounded{\includegraphics[keepaspectratio]{images/Adiyatma_23030630062__EMT4Geometry-002.png}}
\caption{images/Adiyatma\_23030630062\_\_EMT4Geometry-002.png}
\end{figure}

Tampilkan koordinat titik pusat dan jari-jari lingkaran luar.

\textgreater O, R

\begin{verbatim}
[1.16667,  1.16667]
1.17851130198
\end{verbatim}

Sekarang akan digambar lingkaran dalam segitiga ABC. Titik pusat lingkaran dalam adalah titik potong garis-garis bagi sudut.

\textgreater l=angleBisector(A,C,B); // garis bagi \textless ACB

\textgreater g=angleBisector(C,A,B); // garis bagi \textless CAB

\textgreater P=lineIntersection(l,g) // titik potong kedua garis bagi sudut

\begin{verbatim}
[0.86038,  0.86038]
\end{verbatim}

Tambahkan semuanya ke plot.

\textgreater color(5); plotLine(l); plotLine(g); color(1); // gambar kedua garis bagi sudut

\textgreater plotPoint(P,``P''); // gambar titik potongnya

\textgreater r=norm(P-projectToLine(P,lineThrough(A,B))) // jari-jari lingkaran dalam

\begin{verbatim}
0.509653732104
\end{verbatim}

\textgreater plotCircle(circleWithCenter(P,r),``Lingkaran dalam segitiga ABC''): // gambar lingkaran dalam

\begin{figure}
\centering
\pandocbounded{\includegraphics[keepaspectratio]{images/Adiyatma_23030630062__EMT4Geometry-003.png}}
\caption{images/Adiyatma\_23030630062\_\_EMT4Geometry-003.png}
\end{figure}

\section{Latihan}\label{latihan}

\begin{enumerate}
\def\labelenumi{\arabic{enumi}.}
\tightlist
\item
  Tentukan ketiga titik singgung lingkaran dalam dengan sisi-sisi segitiga ABC.
\end{enumerate}

\textgreater setPlotRange(-2.5,4.5,-2.5,4.5);

\textgreater A={[}-2,1{]}; plotPoint(A,``A'');

\textgreater B={[}1,-2{]}; plotPoint(B,``B'');

\textgreater C={[}4,4{]}; plotPoint(C,``C'');

\begin{enumerate}
\def\labelenumi{\arabic{enumi}.}
\setcounter{enumi}{1}
\tightlist
\item
  Gambar segitiga dengan titik-titik sudut ketiga titik singgung tersebut.
\end{enumerate}

\textgreater plotSegment(A,B,``c'')

\textgreater plotSegment(B,C,``a'')

\textgreater plotSegment(A,C,``b'')

\textgreater aspect(1):

\begin{figure}
\centering
\pandocbounded{\includegraphics[keepaspectratio]{images/Adiyatma_23030630062__EMT4Geometry-004.png}}
\caption{images/Adiyatma\_23030630062\_\_EMT4Geometry-004.png}
\end{figure}

\begin{enumerate}
\def\labelenumi{\arabic{enumi}.}
\setcounter{enumi}{2}
\tightlist
\item
  Tunjukkan bahwa garis bagi sudut yang ke tiga juga melalui titik pusat lingkaran dalam.
\end{enumerate}

\textgreater l=angleBisector(A,C,B);

\textgreater g=angleBisector(C,A,B);

\textgreater P=lineIntersection(l,g)

\begin{verbatim}
[0.581139,  0.581139]
\end{verbatim}

\textgreater color(5); plotLine(l); plotLine(g); color(1);

\textgreater plotPoint(P,``P'');

\textgreater r=norm(P-projectToLine(P,lineThrough(A,B)))

\begin{verbatim}
1.52896119631
\end{verbatim}

\textgreater plotCircle(circleWithCenter(P,r),``Lingkaran dalam segitiga ABC''):

\begin{figure}
\centering
\pandocbounded{\includegraphics[keepaspectratio]{images/Adiyatma_23030630062__EMT4Geometry-005.png}}
\caption{images/Adiyatma\_23030630062\_\_EMT4Geometry-005.png}
\end{figure}

Jadi, terbukti bahwa garis bagi sudut yang ketiga juga melalui titik pusat lingkaran dalam.

\begin{enumerate}
\def\labelenumi{\arabic{enumi}.}
\setcounter{enumi}{3}
\tightlist
\item
  Gambar jari-jari lingkaran dalam.
\end{enumerate}

\textgreater r=norm(P-projectToLine(P,lineThrough(A,B)))

\begin{verbatim}
1.52896119631
\end{verbatim}

\textgreater plotCircle(circleWithCenter(P,r),``Lingkaran dalam segitiga ABC''):

\begin{figure}
\centering
\pandocbounded{\includegraphics[keepaspectratio]{images/Adiyatma_23030630062__EMT4Geometry-006.png}}
\caption{images/Adiyatma\_23030630062\_\_EMT4Geometry-006.png}
\end{figure}

\section{Contoh 2: Geometri Simbolik}\label{contoh-2-geometri-simbolik}

Kita dapat menghitung geometri eksak dan simbolik menggunakan Maxima.

File geometri.e menyediakan fungsi yang sama (dan lebih banyak lagi) di Maxima. Namun, kita dapat menggunakan perhitungan simbolis sekarang.

\textgreater A \&= {[}1,0{]}; B \&= {[}0,1{]}; C \&= {[}2,2{]}; // menentukan tiga titik A, B, C

Fungsi untuk garis dan lingkaran bekerja seperti fungsi Euler, tetapi memberikan perhitungan simbolis.

\textgreater c \&= lineThrough(B,C) // c=BC

\begin{verbatim}
                             [- 1, 2, 2]
\end{verbatim}

Kita bisa mendapatkan persamaan garis dengan mudah.

\textgreater\$getLineEquation(c,x,y), \$solve(\%,y) \textbar{} expand // persamaan garis c

\[2\,y-x=2\] \[\left[ y=\frac{x}{2}+1 \right]\]\textgreater\$getLineEquation(lineThrough(A,{[}x1,y1{]}),x,y) // persamaan garis melalui A dan (x1, y1)

\[\left({\it x_1}-1\right)\,y-x\,{\it y_1}=-{\it y_1}\]\textgreater h \&= perpendicular(A,lineThrough(B,C)) // h melalui A tegak lurus BC

\begin{verbatim}
                              [2, 1, 2]
\end{verbatim}

\textgreater Q \&= lineIntersection(c,h) // Q titik potong garis c=BC dan h

\begin{verbatim}
                                 2  6
                                [-, -]
                                 5  5
\end{verbatim}

\textgreater\$projectToLine(A,lineThrough(B,C)) // proyeksi A pada BC

\[\left[ \frac{2}{5} , \frac{6}{5} \right]\]\textgreater\$distance(A,Q) // jarak AQ

\[\frac{3}{\sqrt{5}}\]\textgreater cc \&= circleThrough(A,B,C); \$cc // (titik pusat dan jari-jari) lingkaran melalui A, B, C

\[\left[ \frac{7}{6} , \frac{7}{6} , \frac{5}{3\,\sqrt{2}} \right]\]\textgreater r\&=getCircleRadius(cc); \$r , \$float(r) // tampilkan nilai jari-jari

\[\frac{5}{3\,\sqrt{2}}\] \[1.178511301977579\]\textgreater\$computeAngle(A,C,B) // nilai \textless ACB

\[\arccos \left(\frac{4}{5}\right)\]\textgreater\$solve(getLineEquation(angleBisector(A,C,B),x,y),y){[}1{]} // persamaan garis bagi \textless ACB

\[y=x\]\textgreater P \&= lineIntersection(angleBisector(A,C,B),angleBisector(C,B,A)); \$P // titik potong 2 garis bagi sudut

\[\left[ \frac{\sqrt{2}\,\sqrt{5}+2}{6} , \frac{\sqrt{2}\,\sqrt{5}+2}{6} \right]\]\textgreater P() // hasilnya sama dengan perhitungan sebelumnya

\begin{verbatim}
[0.86038,  0.86038]
\end{verbatim}

\section{Garis dan Lingkaran yang Berpotongan}\label{garis-dan-lingkaran-yang-berpotongan}

Tentu saja, kita juga dapat memotong garis dengan lingkaran, dan lingkaran dengan lingkaran.

\textgreater A \&:= {[}1,0{]}; c=circleWithCenter(A,4);

\textgreater B \&:= {[}1,2{]}; C \&:= {[}2,1{]}; l=lineThrough(B,C);

\textgreater setPlotRange(5); plotCircle(c); plotLine(l);

Perpotongan garis dengan lingkaran menghasilkan dua titik dan jumlah titik potong.

\textgreater\{P1,P2,f\}=lineCircleIntersections(l,c);

\textgreater P1, P2,

\begin{verbatim}
[4.64575,  -1.64575]
[-0.645751,  3.64575]
\end{verbatim}

\textgreater plotPoint(P1); plotPoint(P2):

\begin{figure}
\centering
\pandocbounded{\includegraphics[keepaspectratio]{images/Adiyatma_23030630062__EMT4Geometry-018.png}}
\caption{images/Adiyatma\_23030630062\_\_EMT4Geometry-018.png}
\end{figure}

Begitu pula di Maxima.

\textgreater c \&= circleWithCenter(A,4) // lingkaran dengan pusat A jari-jari 4

\begin{verbatim}
                              [1, 0, 4]
\end{verbatim}

\textgreater l \&= lineThrough(B,C) // garis l melalui B dan C

\begin{verbatim}
                              [1, 1, 3]
\end{verbatim}

\textgreater\$lineCircleIntersections(l,c) \textbar{} radcan, // titik potong lingkaran c dan garis l

\[\left[ \left[ \sqrt{7}+2 , 1-\sqrt{7} \right]  , \left[ 2-\sqrt{7} , \sqrt{7}+1 \right]  \right]\]Akan ditunjukkan bahwa sudut-sudut yang menghadap bsuusr yang sama adalah sama besar.

\textgreater C=A+normalize({[}-2,-3{]})*4; plotPoint(C); plotSegment(P1,C); plotSegment(P2,C);

\textgreater degprint(computeAngle(P1,C,P2))

\begin{verbatim}
69°17'42.68''
\end{verbatim}

\textgreater C=A+normalize({[}-4,-3{]})*4; plotPoint(C); plotSegment(P1,C); plotSegment(P2,C);

\textgreater degprint(computeAngle(P1,C,P2))

\begin{verbatim}
69°17'42.68''
\end{verbatim}

\textgreater insimg;

\begin{figure}
\centering
\pandocbounded{\includegraphics[keepaspectratio]{images/Adiyatma_23030630062__EMT4Geometry-020.png}}
\caption{images/Adiyatma\_23030630062\_\_EMT4Geometry-020.png}
\end{figure}

\section{Garis Sumbu}\label{garis-sumbu}

Berikut adalah langkah-langkah menggambar garis sumbu ruas garis AB:

\begin{enumerate}
\def\labelenumi{\arabic{enumi}.}
\item
  Gambar lingkaran dengan pusat A melalui B.
\item
  Gambar lingkaran dengan pusat B melalui A.
\item
  Tarik garis melallui kedua titik potong kedua lingkaran tersebut. Garis ini merupakan garis sumbu (melalui titik tengah dan tegak lurus) AB.
\end{enumerate}

\textgreater A={[}2,2{]}; B={[}-1,-2{]};

\textgreater c1=circleWithCenter(A,distance(A,B));

\textgreater c2=circleWithCenter(B,distance(A,B));

\textgreater\{P1,P2,f\}=circleCircleIntersections(c1,c2);

\textgreater l=lineThrough(P1,P2);

\textgreater setPlotRange(5); plotCircle(c1); plotCircle(c2);

\textgreater plotPoint(A); plotPoint(B); plotSegment(A,B); plotLine(l):

\begin{figure}
\centering
\pandocbounded{\includegraphics[keepaspectratio]{images/Adiyatma_23030630062__EMT4Geometry-021.png}}
\caption{images/Adiyatma\_23030630062\_\_EMT4Geometry-021.png}
\end{figure}

Selanjutnya, kami melakukan hal yang sama di Maxima dengan koordinat umum.

\textgreater A \&= {[}a1,a2{]}; B \&= {[}b1,b2{]};

\textgreater c1 \&= circleWithCenter(A,distance(A,B));

\textgreater c2 \&= circleWithCenter(B,distance(A,B));

\textgreater P \&= circleCircleIntersections(c1,c2); P1 \&= P{[}1{]}; P2 \&= P{[}2{]};

Persamaan untuk persimpangan cukup terlibat. Tetapi kita dapat menyederhanakannya, jika kita memecahkan y.

\textgreater g \&= getLineEquation(lineThrough(P1,P2),x,y);

\textgreater\$solve(g,y)

\[\left[ y=\frac{-\left(2\,{\it b_1}-2\,{\it a_1}\right)\,x+{\it b_2} ^2+{\it b_1}^2-{\it a_2}^2-{\it a_1}^2}{2\,{\it b_2}-2\,{\it a_2}}\right]\]Ini memang sama dengan tegak lurus tengah, yang dihitung dengan cara yang sama sekali berbeda.

\textgreater\$solve(getLineEquation(middlePerpendicular(A,B),x,y),y)

\[\left[ y=\frac{-\left(2\,{\it b_1}-2\,{\it a_1}\right)\,x+{\it b_2} ^2+{\it b_1}^2-{\it a_2}^2-{\it a_1}^2}{2\,{\it b_2}-2\,{\it a_2}}  \right]\]\textgreater h \&=getLineEquation(lineThrough(A,B),x,y);

\textgreater\$solve(h,y)

\[\left[ y=\frac{\left({\it b_2}-{\it a_2}\right)\,x-{\it a_1}\, {\it b_2}+{\it a_2}\,{\it b_1}}{{\it b_1}-{\it a_1}} \right]\]Perhatikan hasil kali gradien garis g dan h adalah:

Artinya kedua garis tegak lurus.

\chapter{Contoh 3: Rumus Heron}\label{contoh-3-rumus-heron}

Rumus Heron menyatakan bahwa luas segitiga dengan panjang sisi-sisi a, b dan c adalah:

Untuk membuktikan hal ini kita misalkan C(0,0), B(a,0) dan A(x,y), b=AC, c=AB. Luas segitiga ABC adalah

Nilai y didapat dengan menyelesaikan sistem persamaan:

\textgreater sol \&= solve({[}x\textsuperscript{2+y}2=b\textsuperscript{2,(x-a)}2+y\textsuperscript{2=c}2{]},{[}x,y{]})

\begin{verbatim}
                                  []
\end{verbatim}

Ekstrak solusi y.

\textgreater ysol \&= y with sol{[}2{]}{[}2{]}; \$ysol

\begin{verbatim}
Maxima said:
part: invalid index of list or matrix.
 -- an error. To debug this try: debugmode(true);

Error in:
ysol &amp;= y with sol[2][2]; $ysol ...
                        ^
\end{verbatim}

Kami mendapatkan rumus Heron.

\textgreater function H(a,b,c) \&= sqrt(factor((ysol*a/2)\^{}2)); \$'H(a,b,c)=H(a,b,c)

\[H\left(a , b , \left[ 1 , 0 , 4 \right] \right)=\frac{\left| a \right| \,\left| {\it ysol}\right| }{2}\]Tentu saja, setiap segitiga persegi panjang adalah kasus yang terkenal.

\textgreater H(3,4,5) //luas segitiga siku-siku dengan panjang sisi 3, 4, 5

\begin{verbatim}
Variable or function ysol not found.
Try "trace errors" to inspect local variables after errors.
H:
    useglobal; return abs(a)*abs(ysol)/2 
Error in:
H(3,4,5) //luas segitiga siku-siku dengan panjang sisi 3, 4, 5 ...
        ^
\end{verbatim}

Dan juga jelas, bahwa ini adalah segitiga dengan luas maksimal dan dua sisi 3 dan 4.

\textgreater aspect (1.5); plot2d(\&H(3,4,x),1,7): // Kurva luas segitiga sengan panjang sisi 3, 4, x (1\textless= x \textless=7)

\begin{verbatim}
Variable or function ysol not found.
Error in expression: 3*abs(ysol)/2
%ploteval:
    y0=f$(x[1],args());
adaptiveevalone:
    s=%ploteval(g$,t;args());
Try "trace errors" to inspect local variables after errors.
plot2d:
    dw/n,dw/n^2,dw/n,auto;args());
\end{verbatim}

Kasus umum juga berfungsi.

\textgreater\$solve(diff(H(a,b,c)\^{}2,c)=0,c)

\begin{verbatim}
Maxima said:
diff: second argument must be a variable; found [1,0,4]
 -- an error. To debug this try: debugmode(true);

Error in:
$solve(diff(H(a,b,c)^2,c)=0,c) ...
                              ^
\end{verbatim}

Sekarang mari kita cari himpunan semua titik di mana b+c=d untuk beberapa konstanta d.~Diketahui bahwa ini adalah elips.

\textgreater s1 \&= subst(d-c,b,sol{[}2{]}); \$s1

\begin{verbatim}
Maxima said:
part: invalid index of list or matrix.
 -- an error. To debug this try: debugmode(true);

Error in:
s1 &amp;= subst(d-c,b,sol[2]); $s1 ...
                         ^
\end{verbatim}

Dan buat fungsi ini.

\textgreater function fx(a,c,d) \&= rhs(s1{[}1{]}); \$fx(a,c,d), function fy(a,c,d) \&= rhs(s1{[}2{]}); \$fy(a,c,d)

\[0\] \[0\]Sekarang kita bisa menggambar setnya. Sisi b bervariasi dari 1 hingga 4. Diketahui bahwa kita mendapatkan elips.

\textgreater aspect(1); plot2d(\&fx(3,x,5),\&fy(3,x,5),xmin=1,xmax=4,square=1):

\begin{figure}
\centering
\pandocbounded{\includegraphics[keepaspectratio]{images/Adiyatma_23030630062__EMT4Geometry-028.png}}
\caption{images/Adiyatma\_23030630062\_\_EMT4Geometry-028.png}
\end{figure}

Kita dapat memeriksa persamaan umum untuk elips ini, yaitu.

di mana (xm,ym) adalah pusat, dan u dan v adalah setengah sumbu.

\textgreater\$ratsimp((fx(a,c,d)-a/2)\textsuperscript{2/u}2+fy(a,c,d)\textsuperscript{2/v}2 with {[}u=d/2,v=sqrt(d\textsuperscript{2-a}2)/2{]})

\[\frac{a^2}{d^2}\]Kita lihat bahwa tinggi dan luas segitiga adalah maksimal untuk x=0. Jadi luas segitiga dengan a+b+c=d maksimal jika segitiga sama sisi. Kami ingin menurunkan ini secara analitis.

\textgreater eqns \&= {[}diff(H(a,b,d-(a+b))\textsuperscript{2,a)=0,diff(H(a,b,d-(a+b))}2,b)=0{]}; \$eqns

\[\left[ \frac{a\,{\it ysol}^2}{2}=0 , 0=0 \right]\]Kami mendapatkan beberapa minima, yang termasuk dalam segitiga dengan satu sisi 0, dan solusinya a=b=c=d/3.

\textgreater\$solve(eqns,{[}a,b{]})

\[\left[ \left[ a=0 , b={\it \%r_1} \right]  \right]\]Ada juga metode Lagrange, memaksimalkan H(a,b,c)\^{}2 terhadap a+b+d=d.

\textgreater\&solve({[}diff(H(a,b,c)\textsuperscript{2,a)=la,diff(H(a,b,c)}2,b)=la, \ldots{}\\
\textgreater{} diff(H(a,b,c)\^{}2,c)=la,a+b+c=d{]},{[}a,b,c,la{]})

\begin{verbatim}
Maxima said:
diff: second argument must be a variable; found [1,0,4]
 -- an error. To debug this try: debugmode(true);

Error in:
... la,    diff(H(a,b,c)^2,c)=la,a+b+c=d],[a,b,c,la]) ...
                                                     ^
\end{verbatim}

Kita bisa membuat plot situasinya

Pertama-tama atur poin di Maxima.

\textgreater A \&= at({[}x,y{]},sol{[}2{]}); \$A

\begin{verbatim}
Maxima said:
part: invalid index of list or matrix.
 -- an error. To debug this try: debugmode(true);

Error in:
A &amp;= at([x,y],sol[2]); $A ...
                     ^
\end{verbatim}

\textgreater B \&= {[}0,0{]}; \$B, C \&= {[}a,0{]}; \$C

\[\left[ 0 , 0 \right] \]\[\left[ a , 0 \right] \]Kemudian atur rentang plot, dan plot titik-titiknya.

\textgreater setPlotRange(0,5,-2,3); \ldots{}\\
\textgreater{} a=4; b=3; c=2; \ldots{}\\
\textgreater{} plotPoint(mxmeval(``B''),``B''); plotPoint(mxmeval(``C''),``C''); \ldots{}\\
\textgreater{} plotPoint(mxmeval(``A''),``A''):

\begin{verbatim}
Variable a1 not found!
Use global variables or parameters for string evaluation.
Error in Evaluate, superfluous characters found.
Try "trace errors" to inspect local variables after errors.
mxmeval:
    return evaluate(mxm(s));
Error in:
... otPoint(mxmeval("C"),"C"); plotPoint(mxmeval("A"),"A"): ...
                                                     ^
\end{verbatim}

Plot segmen.

\textgreater plotSegment(mxmeval(``A''),mxmeval(``C'')); \ldots{}\\
\textgreater{} plotSegment(mxmeval(``B''),mxmeval(``C'')); \ldots{}\\
\textgreater{} plotSegment(mxmeval(``B''),mxmeval(``A'')):

\begin{verbatim}
Variable a1 not found!
Use global variables or parameters for string evaluation.
Error in Evaluate, superfluous characters found.
Try "trace errors" to inspect local variables after errors.
mxmeval:
    return evaluate(mxm(s));
Error in:
plotSegment(mxmeval("A"),mxmeval("C")); plotSegment(mxmeval("B ...
                        ^
\end{verbatim}

Hitung tegak lurus tengah di Maxima.

\textgreater h \&= middlePerpendicular(A,B); g \&= middlePerpendicular(B,C);

Dan pusat lingkaran.

\textgreater U \&= lineIntersection(h,g);

Kami mendapatkan rumus untuk jari-jari lingkaran.

\textgreater\&assume(a\textgreater0,b\textgreater0,c\textgreater0); \$distance(U,B) \textbar{} radcan

\[\frac{\sqrt{{\it a_2}^2+{\it a_1}^2}\,\sqrt{{\it a_2}^2+{\it a_1}^2 -2\,a\,{\it a_1}+a^2}}{2\,\left| {\it a_2}\right| }\]Mari kita tambahkan ini ke plot.

\textgreater plotPoint(U()); \ldots{}\\
\textgreater{} plotCircle(circleWithCenter(mxmeval(``U''),mxmeval(``distance(U,C)''))):

\begin{verbatim}
Variable a2 not found!
Use global variables or parameters for string evaluation.
Error in ^
Error in expression: [a/2,(a2^2+a1^2-a*a1)/(2*a2)]
Error in:
plotPoint(U()); plotCircle(circleWithCenter(mxmeval("U"),mxmev ...
             ^
\end{verbatim}

Menggunakan geometri, kami memperoleh rumus sederhana

untuk radiusnya. Kami dapat memeriksa, apakah ini benar dengan Maxima. Maxima akan memfaktorkan ini hanya jika kita kuadratkan.

\textgreater\$c\textsuperscript{2/sin(computeAngle(A,B,C))}2 \textbar{} factor

\[\left[ \frac{{\it a_2}^2+{\it a_1}^2}{{\it a_2}^2} , 0 , \frac{16\,\left({\it a_2}^2+{\it a_1}^2\right)}{{\it a_2}^2} \right]\]\#\# Contoh 4: Garis Euler dan Parabola

Garis Euler adalah garis yang ditentukan dari sembarang segitiga yang tidak sama sisi. Ini adalah garis tengah segitiga, dan melewati beberapa titik penting yang ditentukan dari segitiga, termasuk orthocenter, circumcenter, centroid, titik Exeter dan pusat lingkaran sembilan titik segitiga.

Untuk demonstrasi, kami menghitung dan memplot garis Euler dalam sebuah segitiga.

Pertama, kita mendefinisikan sudut-sudut segitiga di Euler. Kami menggunakan definisi, yang terlihat dalam ekspresi simbolis.

\textgreater A::={[}-1,-1{]}; B::={[}2,0{]}; C::={[}1,2{]};

Untuk memplot objek geometris, kami menyiapkan area plot, dan menambahkan titik ke sana. Semua plot objek geometris ditambahkan ke plot saat ini.

\textgreater setPlotRange(3); plotPoint(A,``A''); plotPoint(B,``B''); plotPoint(C,``C'');

Kita juga bisa menambahkan sisi segitiga.

\textgreater plotSegment(A,B,``\,``); plotSegment(B,C,''``); plotSegment(C,A,''\,``):

\begin{figure}
\centering
\pandocbounded{\includegraphics[keepaspectratio]{images/Adiyatma_23030630062__EMT4Geometry-036.png}}
\caption{images/Adiyatma\_23030630062\_\_EMT4Geometry-036.png}
\end{figure}

Berikut adalah luas segitiga, menggunakan rumus determinan. Tentu saja, kita harus mengambil nilai absolut dari hasil ini.

\textgreater\$areaTriangle(A,B,C)

\[-\frac{7}{2}\]Kita dapat menghitung koefisien sisi c.

\textgreater c \&= lineThrough(A,B)

\begin{verbatim}
                            [- 1, 3, - 2]
\end{verbatim}

Dan juga dapatkan rumus untuk baris ini.

\textgreater\$getLineEquation(c,x,y)

\[3\,y-x=-2\]Untuk bentuk Hesse, kita perlu menentukan sebuah titik, sehingga titik tersebut berada di sisi positif dari bentuk Hesse. Memasukkan titik menghasilkan jarak positif ke garis.

\textgreater\$getHesseForm(c,x,y,C), \$at(\%,{[}x=C{[}1{]},y=C{[}2{]}{]})

\[\frac{3\,y-x+2}{\sqrt{10}}\] \[\frac{7}{\sqrt{10}}\]Sekarang kita hitung lingkaran luar ABC.

\textgreater LL \&= circleThrough(A,B,C); \$getCircleEquation(LL,x,y)

\[\left(y-\frac{5}{14}\right)^2+\left(x-\frac{3}{14}\right)^2=\frac{325}{98}\]\textgreater O \&= getCircleCenter(LL); \$O

\[\left[ \frac{3}{14} , \frac{5}{14} \right]\]Gambarkan lingkaran dan pusatnya. Cu dan U adalah simbolis. Kami mengevaluasi ekspresi ini untuk Euler.

\textgreater plotCircle(LL()); plotPoint(O(),``O''):

\begin{figure}
\centering
\pandocbounded{\includegraphics[keepaspectratio]{images/Adiyatma_23030630062__EMT4Geometry-043.png}}
\caption{images/Adiyatma\_23030630062\_\_EMT4Geometry-043.png}
\end{figure}

Kita dapat menghitung perpotongan ketinggian di ABC (orthocenter) secara numerik dengan perintah berikut.

\textgreater H \&= lineIntersection(perpendicular(A,lineThrough(C,B)),\ldots{}\\
\textgreater{} perpendicular(B,lineThrough(A,C))); \$H

\[\left[ \frac{11}{7} , \frac{2}{7} \right]\]Sekarang kita dapat menghitung garis Euler dari segitiga.

\textgreater el \&= lineThrough(H,O); \$getLineEquation(el,x,y)

\[-\frac{19\,y}{14}-\frac{x}{14}=-\frac{1}{2}\]Tambahkan ke plot kami.

\textgreater plotPoint(H(),``H''); plotLine(el(),``Garis Euler''):

\begin{figure}
\centering
\pandocbounded{\includegraphics[keepaspectratio]{images/Adiyatma_23030630062__EMT4Geometry-046.png}}
\caption{images/Adiyatma\_23030630062\_\_EMT4Geometry-046.png}
\end{figure}

Pusat gravitasi harus berada di garis ini.

\textgreater M \&= (A+B+C)/3; \$getLineEquation(el,x,y) with

\[-\frac{1}{2}=-\frac{1}{2}\]\textgreater plotPoint(M(),``M''): // titik berat

\begin{figure}
\centering
\pandocbounded{\includegraphics[keepaspectratio]{images/Adiyatma_23030630062__EMT4Geometry-048.png}}
\caption{images/Adiyatma\_23030630062\_\_EMT4Geometry-048.png}
\end{figure}

Teorinya memberitahu kita MH=2*MO. Kita perlu menyederhanakan dengan radcan untuk mencapai ini.

\textgreater\$distance(M,H)/distance(M,O)\textbar radcan

\[2\]Fungsi termasuk fungsi untuk sudut juga.

\textgreater\$computeAngle(A,C,B), degprint(\%())

\[\arccos \left(\frac{4}{\sqrt{5}\,\sqrt{13}}\right)\] 60°15'18.43'\,'

Persamaan untuk pusat incircle tidak terlalu bagus.

\textgreater Q \&= lineIntersection(angleBisector(A,C,B),angleBisector(C,B,A))\textbar radcan; \$Q

\[\left[ \frac{\left(2^{\frac{3}{2}}+1\right)\,\sqrt{5}\,\sqrt{13}-15 \,\sqrt{2}+3}{14} , \frac{\left(\sqrt{2}-3\right)\,\sqrt{5}\,\sqrt{ 13}+5\,2^{\frac{3}{2}}+5}{14} \right]\]Mari kita hitung juga ekspresi untuk jari-jari lingkaran yang tertulis.

\textgreater r \&= distance(Q,projectToLine(Q,lineThrough(A,B)))\textbar ratsimp; \$r

\[\frac{\sqrt{\left(-41\,\sqrt{2}-31\right)\,\sqrt{5}\,\sqrt{13}+115 \,\sqrt{2}+614}}{7\,\sqrt{2}}\]\textgreater LD \&= circleWithCenter(Q,r); // Lingkaran dalam

Mari kita tambahkan ini ke plot.

\textgreater color(5); plotCircle(LD()):

\begin{figure}
\centering
\pandocbounded{\includegraphics[keepaspectratio]{images/Adiyatma_23030630062__EMT4Geometry-053.png}}
\caption{images/Adiyatma\_23030630062\_\_EMT4Geometry-053.png}
\end{figure}

\section{Parabola}\label{parabola}

Selanjutnya akan dicari persamaan tempat kedudukan titik-titik yang berjarak sama ke titik C dan ke garis AB.

\textgreater p \&= getHesseForm(lineThrough(A,B),x,y,C)-distance({[}x,y{]},C); \$p='0

\[\frac{3\,y-x+2}{\sqrt{10}}-\sqrt{\left(2-y\right)^2+\left(1-x\right)^2}=0\]Persamaan tersebut dapat digambar menjadi satu dengan gambar sebelumnya.

\textgreater plot2d(p,level=0,add=1,contourcolor=6):

\begin{figure}
\centering
\pandocbounded{\includegraphics[keepaspectratio]{images/Adiyatma_23030630062__EMT4Geometry-055.png}}
\caption{images/Adiyatma\_23030630062\_\_EMT4Geometry-055.png}
\end{figure}

Ini seharusnya menjadi beberapa fungsi, tetapi pemecah default Maxima hanya dapat menemukan solusinya, jika kita kuadratkan persamaannya. Akibatnya, kami mendapatkan solusi palsu.

\textgreater akar \&= solve(getHesseForm(lineThrough(A,B),x,y,C)\textsuperscript{2-distance({[}x,y{]},C)}2,y)

\begin{verbatim}
        [y = - 3 x - sqrt(70) sqrt(9 - 2 x) + 26, 
                              y = - 3 x + sqrt(70) sqrt(9 - 2 x) + 26]
\end{verbatim}

Solusi pertama adalah

maxima: akar{[}1{]}

Menambahkan solusi pertama ke plot menunjukkan, bahwa itu memang jalan yang kita cari. Teorinya memberi tahu kita bahwa itu adalah parabola yang diputar.

\textgreater plot2d(\&rhs(akar{[}1{]}),add=1):

\begin{figure}
\centering
\pandocbounded{\includegraphics[keepaspectratio]{images/Adiyatma_23030630062__EMT4Geometry-056.png}}
\caption{images/Adiyatma\_23030630062\_\_EMT4Geometry-056.png}
\end{figure}

\textgreater function g(x) \&= rhs(akar{[}1{]}); \$'g(x)= g(x)// fungsi yang mendefinisikan kurva di atas

\[g\left(x\right)=-3\,x-\sqrt{70}\,\sqrt{9-2\,x}+26\]\textgreater T \&={[}-1, g(-1){]}; // ambil sebarang titik pada kurva tersebut

\textgreater dTC \&= distance(T,C); \$fullratsimp(dTC), \$float(\%) // jarak T ke C

\[\sqrt{1503-54\,\sqrt{11}\,\sqrt{70}}\] \[2.135605779339061\]\textgreater U \&= projectToLine(T,lineThrough(A,B)); \$U // proyeksi T pada garis AB

\[\left[ \frac{80-3\,\sqrt{11}\,\sqrt{70}}{10} , \frac{20-\sqrt{11}\,\sqrt{70}}{10} \right]\]\textgreater dU2AB \&= distance(T,U); \$fullratsimp(dU2AB), \$float(\%) // jatak T ke AB

\[\sqrt{1503-54\,\sqrt{11}\,\sqrt{70}}\] \[2.135605779339061\]Ternyata jarak T ke C sama dengan jarak T ke AB. Coba Anda pilih titik T yang lain dan ulangi perhitungan-perhitungan di atas untuk menunjukkan bahwa hasilnya juga sama.

\section{Contoh 5: Trigonometri Rasional}\label{contoh-5-trigonometri-rasional}

Ini terinspirasi dari ceramah N.J.Wildberger. Dalam bukunya ``Divine Proportions'', Wildberger mengusulkan untuk mengganti pengertian klasik tentang jarak dan sudut dengan kuadrat dan penyebaran. Dengan menggunakan ini, memang mungkin untuk menghindari fungsi trigonometri dalam banyak contoh, dan tetap ``rasional''.

Berikut ini, saya memperkenalkan konsep, dan memecahkan beberapa masalah. Saya menggunakan perhitungan simbolik Maxima di sini, yang menyembunyikan keuntungan utama dari trigonometri rasional bahwa perhitungan hanya dapat dilakukan dengan kertas dan pensil. Anda diundang untuk memeriksa hasil tanpa komputer.

Intinya adalah bahwa perhitungan rasional simbolis sering kali menghasilkan hasil yang sederhana. Sebaliknya, trigonometri klasik menghasilkan hasil trigonometri yang rumit, yang hanya mengevaluasi perkiraan numerik.

\textgreater load geometry;

Untuk pengenalan pertama, kami menggunakan segitiga persegi panjang dengan proporsi Mesir terkenal 3, 4 dan 5. Perintah berikut adalah perintah Euler untuk merencanakan geometri bidang yang terdapat dalam file Euler ``geometry.e''.

\textgreater C\&:={[}0,0{]}; A\&:={[}4,0{]}; B\&:={[}0,3{]}; \ldots{}\\
\textgreater{} setPlotRange(-1,5,-1,5); \ldots{}\\
\textgreater{} plotPoint(A,``A''); plotPoint(B,``B''); plotPoint(C,``C''); \ldots{}\\
\textgreater{} plotSegment(B,A,``c''); plotSegment(A,C,``b''); plotSegment(C,B,``a''); \ldots{}\\
\textgreater{} insimg(30);

\begin{figure}
\centering
\pandocbounded{\includegraphics[keepaspectratio]{images/Adiyatma_23030630062__EMT4Geometry-063.png}}
\caption{images/Adiyatma\_23030630062\_\_EMT4Geometry-063.png}
\end{figure}

Tentu saja,

di mana wa adalah sudut di A. Cara yang biasa untuk menghitung sudut ini, adalah dengan mengambil invers dari fungsi sinus. Hasilnya adalah sudut yang tidak dapat dicerna, yang hanya dapat dicetak kira-kira.

\textgreater wa := arcsin(3/5); degprint(wa)

\begin{verbatim}
36°52'11.63''
\end{verbatim}

Trigonometri rasional mencoba menghindari hal ini.

Gagasan pertama trigonometri rasional adalah kuadran, yang menggantikan jarak. Sebenarnya, itu hanya jarak kuadrat. Berikut ini, a, b, dan c menunjukkan kuadrat dari sisi-sisinya.

Teorema Pythogoras menjadi a+b=c.

\textgreater a \&= 3\^{}2; b \&= 4\^{}2; c \&= 5\^{}2; \&a+b=c

\begin{verbatim}
                               25 = 25
\end{verbatim}

Pengertian kedua dari trigonometri rasional adalah penyebaran. Spread mengukur pembukaan antar baris. Ini adalah 0, jika garis-garisnya sejajar, dan 1, jika garis-garisnya persegi panjang. Ini adalah kuadrat sinus sudut antara dua garis.

Penyebaran garis AB dan AC pada gambar di atas didefinisikan sebagai:

di mana a dan c adalah kuadrat dari sembarang segitiga siku-siku dengan salah satu sudut di A.

\textgreater sa \&= a/c; \$sa

\[\frac{9}{25}\]Ini lebih mudah dihitung daripada sudut, tentu saja. Tetapi Anda kehilangan properti bahwa sudut dapat ditambahkan dengan mudah.

Tentu saja, kita dapat mengonversi nilai perkiraan untuk sudut wa menjadi sprad, dan mencetaknya sebagai pecahan.

\textgreater fracprint(sin(wa)\^{}2)

\begin{verbatim}
9/25
\end{verbatim}

Hukum kosinus trgonometri klasik diterjemahkan menjadi ``hukum silang'' berikut.

Di sini a, b, dan c adalah kuadrat dari sisi-sisi segitiga, dan sa adalah penyebaran sudut A. Sisi a, seperti biasa, berhadapan dengan sudut A.

Hukum ini diimplementasikan dalam file geometri.e yang kami muat ke Euler.

\textgreater\$crosslaw(aa,bb,cc,saa)

\[\left[ \left({\it bb}-{\it aa}+\frac{7}{6}\right)^2 , \left({\it bb}-{\it aa}+\frac{7}{6}\right)^2 , \left({\it bb}-{\it aa}+ \frac{5}{3\,\sqrt{2}}\right)^2 \right] =\left[ \frac{14\,{\it bb}\, \left(1-{\it saa}\right)}{3} , \frac{14\,{\it bb}\,\left(1-{\it saa} \right)}{3} , \frac{5\,2^{\frac{3}{2}}\,{\it bb}\,\left(1-{\it saa}\right)}{3} \right]\]Dalam kasus kami, kami mendapatkan

\textgreater\$crosslaw(a,b,c,sa)

\[1024=1024\]Mari kita gunakan crosslaw ini untuk mencari spread di A. Untuk melakukan ini, kita buat crosslaw untuk kuadran a, b, dan c, dan selesaikan untuk spread yang tidak diketahui sa.

Anda dapat melakukannya dengan tangan dengan mudah, tetapi saya menggunakan Maxima. Tentu saja, kami mendapatkan hasilnya, kami sudah memilikinya.

\textgreater\$crosslaw(a,b,c,x), \$solve(\%,x)

\[1024=1600\,\left(1-x\right)\] \[\left[ x=\frac{9}{25} \right]\]Kita sudah tahu ini. Definisi spread adalah kasus khusus dari crosslaw.

Kita juga dapat menyelesaikan ini untuk umum a,b,c.~Hasilnya adalah rumus yang menghitung penyebaran sudut segitiga yang diberikan kuadrat dari ketiga sisinya.

\textgreater\$solve(crosslaw(aa,bb,cc,x),x)

\[\left[ \left[ \frac{168\,{\it bb}\,x+36\,{\it bb}^2+\left(-72\, {\it aa}-84\right)\,{\it bb}+36\,{\it aa}^2-84\,{\it aa}+49}{36} ,  \frac{168\,{\it bb}\,x+36\,{\it bb}^2+\left(-72\,{\it aa}-84\right) \,{\it bb}+36\,{\it aa}^2-84\,{\it aa}+49}{36} , \frac{15\,2^{\frac{ 5}{2}}\,{\it bb}\,x+18\,{\it bb}^2+\left(-36\,{\it aa}-15\,2^{\frac{ 3}{2}}\right)\,{\it bb}+18\,{\it aa}^2-15\,2^{\frac{3}{2}}\,{\it aa} +25}{18} \right] =0 \right]\]Kita bisa membuat fungsi dari hasilnya. Fungsi seperti itu sudah didefinisikan dalam file geometri.e dari Euler.

\textgreater\$spread(a,b,c)

\[\frac{9}{25}\]Sebagai contoh, kita dapat menggunakannya untuk menghitung sudut segitiga dengan sisi

Hasilnya rasional, yang tidak begitu mudah didapat jika kita menggunakan trigonometri klasik.

\textgreater\$spread(a,a,4*a/7)

\[\frac{6}{7}\]Ini adalah sudut dalam derajat.

\textgreater degprint(arcsin(sqrt(6/7)))

\begin{verbatim}
67°47'32.44''
\end{verbatim}

\section{Contoh lain}\label{contoh-lain}

Sekarang, mari kita coba contoh yang lebih maju.

Kami mengatur tiga sudut segitiga sebagai berikut.

\textgreater A\&:={[}1,2{]}; B\&:={[}4,3{]}; C\&:={[}0,4{]}; \ldots{}\\
\textgreater{} setPlotRange(-1,5,1,7); \ldots{}\\
\textgreater{} plotPoint(A,``A''); plotPoint(B,``B''); plotPoint(C,``C''); \ldots{}\\
\textgreater{} plotSegment(B,A,``c''); plotSegment(A,C,``b''); plotSegment(C,B,``a''); \ldots{}\\
\textgreater{} insimg;

\begin{figure}
\centering
\pandocbounded{\includegraphics[keepaspectratio]{images/Adiyatma_23030630062__EMT4Geometry-072.png}}
\caption{images/Adiyatma\_23030630062\_\_EMT4Geometry-072.png}
\end{figure}

Menggunakan Pythogoras, mudah untuk menghitung jarak antara dua titik. Saya pertama kali menggunakan jarak fungsi file Euler untuk geometri. Jarak fungsi menggunakan geometri klasik.

\textgreater\$distance(A,B)

\[\sqrt{10}\]Euler juga mengandung fungsi untuk kuadran antara dua titik.

Dalam contoh berikut, karena c+b bukan a, maka segitiga itu bukan persegi panjang.

\textgreater c \&= quad(A,B); \$c, b \&= quad(A,C); \$b, a \&= quad(B,C); \$a,

\[10\] \[5\] \[17\]Pertama, mari kita hitung sudut tradisional. Fungsi computeAngle menggunakan metode biasa berdasarkan hasil kali titik dua vektor. Hasilnya adalah beberapa pendekatan floating point.

\textgreater wb \&= computeAngle(A,B,C); \$wb, \$(wb/pi*180)()

\[\arccos \left(\frac{11}{\sqrt{10}\,\sqrt{17}}\right)\] 32.4711922908

Dengan menggunakan pensil dan kertas, kita dapat melakukan hal yang sama dengan hukum silang. Kami memasukkan kuadran a, b, dan c ke dalam hukum silang dan menyelesaikan x.

\textgreater\$crosslaw(a,b,c,x), \$solve(\%,x),

\[4=200\,\left(1-x\right)\] \[\left[ x=\frac{49}{50} \right]\]Yaitu, apa yang dilakukan oleh penyebaran fungsi yang didefinisikan dalam ``geometry.e''.

\textgreater sb \&= spread(b,a,c); \$sb

\[\frac{49}{170}\]Maxima mendapatkan hasil yang sama menggunakan trigonometri biasa, jika kita memaksanya. Itu menyelesaikan istilah sin(arccos(\ldots)) menjadi hasil pecahan. Sebagian besar siswa tidak dapat melakukan ini.

\textgreater\$sin(computeAngle(A,B,C))\^{}2

\[\frac{49}{170}\]Setelah kita memiliki spread di B, kita dapat menghitung tinggi ha di sisi a. Ingat bahwa

Menurut definisi.

\textgreater ha \&= c*sb; \$ha

\[\frac{49}{17}\]Gambar berikut telah dihasilkan dengan program geometri C.a.R., yang dapat menggambar kuadrat dan menyebar.

image: (20) Rational\_Geometry\_CaR.png

Menurut definisi, panjang ha adalah akar kuadrat dari kuadratnya.

\textgreater\$sqrt(ha)

\[\frac{7}{\sqrt{17}}\]Sekarang kita dapat menghitung luas segitiga. Jangan lupa, bahwa kita berhadapan dengan kuadrat!

\textgreater\$sqrt(ha)*sqrt(a)/2

\[\frac{7}{2}\]Rumus determinan biasa menghasilkan hasil yang sama.

\textgreater\$areaTriangle(B,A,C)

\[\frac{7}{2}\]\#\# Rumus Heron

Sekarang, mari kita selesaikan masalah ini secara umum!

\textgreater\&remvalue(a,b,c,sb,ha);

Pertama kita hitung spread di B untuk segitiga dengan sisi a, b, dan c.~Kemudian kita menghitung luas kuadrat (``quadrea''?), faktorkan dengan Maxima, dan kita mendapatkan rumus Heron yang terkenal.

Memang, ini sulit dilakukan dengan pensil dan kertas.

\textgreater\$spread(b\textsuperscript{2,c}2,a\^{}2), \$factor(\%*c\textsuperscript{2*a}2/4)

\[\frac{-c^4-\left(-2\,b^2-2\,a^2\right)\,c^2-b^4+2\,a^2\,b^2-a^4}{4 \,a^2\,c^2}\] \[\frac{\left(-c+b+a\right)\,\left(c-b+a\right)\,\left(c+b-a\right)\,\left(c+b+a\right)}{16}\]\#\# Aturan Triple Spread

Kerugian dari spread adalah mereka tidak lagi hanya menambahkan sudut yang sama.

Namun, tiga spread dari sebuah segitiga memenuhi aturan ``triple spread'' berikut.

\textgreater\&remvalue(sa,sb,sc); \$triplespread(sa,sb,sc)

\[\left({\it sc}+{\it sb}+{\it sa}\right)^2=2\,\left({\it sc}^2+ {\it sb}^2+{\it sa}^2\right)+4\,{\it sa}\,{\it sb}\,{\it sc}\]Aturan ini berlaku untuk setiap tiga sudut yang menambah 180 °.

Sejak menyebar

sama, aturan triple spread juga benar, jika

Karena penyebaran sudut negatif adalah sama, aturan penyebaran rangkap tiga juga berlaku, jika

Misalnya, kita dapat menghitung penyebaran sudut 60°. Ini 3/4. Persamaan memiliki solusi kedua, bagaimanapun, di mana semua spread adalah 0.

\textgreater\$solve(triplespread(x,x,x),x)

\[\left[ x=\frac{3}{4} , x=0 \right]\]Sebaran 90° jelas 1. Jika dua sudut dijumlahkan menjadi 90°, sebarannya menyelesaikan persamaan sebaran rangkap tiga dengan a,b,1. Dengan perhitungan berikut kita mendapatkan a+b=1.

\textgreater\$triplespread(x,y,1), \$solve(\%,x)

\[\left(y+x+1\right)^2=2\,\left(y^2+x^2+1\right)+4\,x\,y\] \[\left[ x=1-y \right]\]Karena sebaran 180°-t sama dengan sebaran t, rumus sebaran rangkap tiga juga berlaku, jika satu sudut adalah jumlah atau selisih dua sudut lainnya.

Jadi kita dapat menemukan penyebaran sudut berlipat ganda. Perhatikan bahwa ada dua solusi lagi. Kami membuat ini fungsi.

\textgreater\$solve(triplespread(a,a,x),x), function doublespread(a) \&= factor(rhs(\%{[}1{]}))

\[\left[ x=4\,a-4\,a^2 , x=0 \right]\]\\
- 4 (a - 1) a

\section{Sudut Pembagi}\label{sudut-pembagi}

Ini situasinya, kita sudah tahu.

\textgreater C\&:={[}0,0{]}; A\&:={[}4,0{]}; B\&:={[}0,3{]}; \ldots{}\\
\textgreater{} setPlotRange(-1,5,-1,5); \ldots{}\\
\textgreater{} plotPoint(A,``A''); plotPoint(B,``B''); plotPoint(C,``C''); \ldots{}\\
\textgreater{} plotSegment(B,A,``c''); plotSegment(A,C,``b''); plotSegment(C,B,``a''); \ldots{}\\
\textgreater{} insimg;

\begin{figure}
\centering
\pandocbounded{\includegraphics[keepaspectratio]{images/Adiyatma_23030630062__EMT4Geometry-093.png}}
\caption{images/Adiyatma\_23030630062\_\_EMT4Geometry-093.png}
\end{figure}

Mari kita hitung panjang garis bagi sudut di A. Tetapi kita ingin menyelesaikannya untuk umum a,b,c.

\textgreater\&remvalue(a,b,c);

Jadi pertama-tama kita hitung penyebaran sudut yang dibagi dua di A, dengan menggunakan rumus sebaran rangkap tiga.

Masalah dengan rumus ini muncul lagi. Ini memiliki dua solusi. Kita harus memilih yang benar. Solusi lainnya mengacu pada sudut terbelah 180 °-wa.

\textgreater\$triplespread(x,x,a/(a+b)), \$solve(\%,x), sa2 \&= rhs(\%{[}1{]}); \$sa2

\[\left(2\,x+\frac{a}{b+a}\right)^2=2\,\left(2\,x^2+\frac{a^2}{\left(b+a\right)^2}\right)+\frac{4\,a\,x^2}{b+a}\] \[\left[ x=\frac{-\sqrt{b}\,\sqrt{b+a}+b+a}{2\,b+2\,a} , x=\frac{\sqrt{b}\,\sqrt{b+a}+b+a}{2\,b+2\,a} \right]\] \[\frac{-\sqrt{b}\,\sqrt{b+a}+b+a}{2\,b+2\,a}\]Mari kita periksa persegi panjang Mesir.

\textgreater\$sa2 with {[}a=3\textsuperscript{2,b=4}2{]}

\[\frac{1}{10}\]Kami dapat mencetak sudut dalam Euler, setelah mentransfer penyebaran ke radian.

\textgreater wa2 := arcsin(sqrt(1/10)); degprint(wa2)

\begin{verbatim}
18°26'5.82''
\end{verbatim}

Titik P adalah perpotongan garis bagi sudut dengan sumbu y.

\textgreater P := {[}0,tan(wa2)*4{]}

\begin{verbatim}
[0,  1.33333]
\end{verbatim}

\textgreater plotPoint(P,``P''); plotSegment(A,P):

\begin{figure}
\centering
\pandocbounded{\includegraphics[keepaspectratio]{images/Adiyatma_23030630062__EMT4Geometry-098.png}}
\caption{images/Adiyatma\_23030630062\_\_EMT4Geometry-098.png}
\end{figure}

Mari kita periksa sudut dalam contoh spesifik kita.

\textgreater computeAngle(C,A,P), computeAngle(P,A,B)

\begin{verbatim}
0.321750554397
0.321750554397
\end{verbatim}

Sekarang kita hitung panjang garis bagi AP.

Kami menggunakan teorema sinus dalam segitiga APC. Teorema ini menyatakan bahwa

berlaku dalam segitiga apa pun. Kuadratkan, itu diterjemahkan ke dalam apa yang disebut ``hukum penyebaran''

di mana a,b,c menunjukkan qudrances.

Karena spread CPA adalah 1-sa2, kita dapatkan darinya bisa/1=b/(1-sa2) dan dapat menghitung bisa (kuadran dari garis-bagi sudut).

\textgreater\&factor(ratsimp(b/(1-sa2))); bisa \&= \%; \$bisa

\[\frac{2\,b\,\left(b+a\right)}{\sqrt{b}\,\sqrt{b+a}+b+a}\]Mari kita periksa rumus ini untuk nilai-nilai Mesir kita.

\textgreater sqrt(mxmeval(``at(bisa,{[}a=3\textsuperscript{2,b=4}2{]})'')), distance(A,P)

\begin{verbatim}
4.21637021356
4.21637021356
\end{verbatim}

Kita juga dapat menghitung P menggunakan rumus spread.

\textgreater py\&=factor(ratsimp(sa2*bisa)); \$py

\[-\frac{b\,\left(\sqrt{b}\,\sqrt{b+a}-b-a\right)}{\sqrt{b}\,\sqrt{b+a}+b+a}\]Nilainya sama dengan yang kita dapatkan dengan rumus trigonometri.

\textgreater sqrt(mxmeval(``at(py,{[}a=3\textsuperscript{2,b=4}2{]})''))

\begin{verbatim}
1.33333333333
\end{verbatim}

\section{Sudut Akord}\label{sudut-akord}

Perhatikan situasi berikut.

\textgreater setPlotRange(1.2); \ldots{}\\
\textgreater{} color(1); plotCircle(circleWithCenter({[}0,0{]},1)); \ldots{}\\
\textgreater{} A:={[}cos(1),sin(1){]}; B:={[}cos(2),sin(2){]}; C:={[}cos(6),sin(6){]}; \ldots{}\\
\textgreater{} plotPoint(A,``A''); plotPoint(B,``B''); plotPoint(C,``C''); \ldots{}\\
\textgreater{} color(3); plotSegment(A,B,``c''); plotSegment(A,C,``b''); plotSegment(C,B,``a''); \ldots{}\\
\textgreater{} color(1); O:={[}0,0{]}; plotPoint(O,``0''); \ldots{}\\
\textgreater{} plotSegment(A,O); plotSegment(B,O); plotSegment(C,O,``r''); \ldots{}\\
\textgreater{} insimg;

\begin{figure}
\centering
\pandocbounded{\includegraphics[keepaspectratio]{images/Adiyatma_23030630062__EMT4Geometry-101.png}}
\caption{images/Adiyatma\_23030630062\_\_EMT4Geometry-101.png}
\end{figure}

Kita dapat menggunakan Maxima untuk menyelesaikan rumus penyebaran rangkap tiga untuk sudut-sudut di pusat O untuk r. Jadi kita mendapatkan rumus untuk jari-jari kuadrat dari pericircle dalam hal kuadrat dari sisi.

Kali ini, Maxima menghasilkan beberapa nol kompleks, yang kita abaikan.

\textgreater\&remvalue(a,b,c,r); // hapus nilai-nilai sebelumnya untuk perhitungan baru

\textgreater rabc \&= rhs(solve(triplespread(spread(b,r,r),spread(a,r,r),spread(c,r,r)),r){[}4{]}); \$rabc

\[-\frac{a\,b\,c}{c^2-2\,b\,c+a\,\left(-2\,c-2\,b\right)+b^2+a^2}\]Kita dapat menjadikannya sebagai fungsi Euler.

\textgreater function periradius(a,b,c) \&= rabc;

Mari kita periksa hasilnya untuk poin A,B,C.

\textgreater a:=quadrance(B,C); b:=quadrance(A,C); c:=quadrance(A,B);

Jari-jarinya memang 1.

\textgreater periradius(a,b,c)

\begin{verbatim}
1
\end{verbatim}

Faktanya, spread CBA hanya bergantung pada b dan c.~Ini adalah teorema sudut chord.

\textgreater\$spread(b,a,c)*rabc \textbar{} ratsimp

\[\frac{b}{4}\]Sebenarnya spreadnya adalah b/(4r), dan kita melihat bahwa sudut chord dari chord b adalah setengah dari sudut pusat.

\textgreater\$doublespread(b/(4*r))-spread(b,r,r) \textbar{} ratsimp

\[0\]\#\# Contoh 6: Jarak Minimal pada Bidang

\section{Catatan awal}\label{catatan-awal}

Fungsi yang, ke titik M di bidang, menetapkan jarak AM antara titik tetap A dan M, memiliki garis level yang agak sederhana: lingkaran berpusat di A.

\textgreater\&remvalue();

\textgreater A={[}-1,-1{]};

\textgreater function d1(x,y):=sqrt((x-A{[}1{]})\textsuperscript{2+(y-A{[}2{]})}2)

\textgreater fcontour(``d1'',xmin=-2,xmax=0,ymin=-2,ymax=0,hue=1, \ldots{}\\
\textgreater{} title=``If you see ellipses, please set your window square''):

\begin{figure}
\centering
\pandocbounded{\includegraphics[keepaspectratio]{images/Adiyatma_23030630062__EMT4Geometry-105.png}}
\caption{images/Adiyatma\_23030630062\_\_EMT4Geometry-105.png}
\end{figure}

dan grafiknya juga agak sederhana: bagian atas kerucut:

\textgreater plot3d(``d1'',xmin=-2,xmax=0,ymin=-2,ymax=0):

\begin{figure}
\centering
\pandocbounded{\includegraphics[keepaspectratio]{images/Adiyatma_23030630062__EMT4Geometry-106.png}}
\caption{images/Adiyatma\_23030630062\_\_EMT4Geometry-106.png}
\end{figure}

Tentu saja minimal 0 dicapai di A.

\section{Poin Dua}\label{poin-dua}

Sekarang kita lihat fungsi MA+MB dimana A dan B adalah dua titik (tetap). Ini adalah ``fakta yang diketahui'' bahwa kurva level adalah elips, titik fokusnya adalah A dan B; kecuali untuk AB minimum yang konstan pada segmen {[}AB{]}:

\textgreater B={[}1,-1{]};

\textgreater function d2(x,y):=d1(x,y)+sqrt((x-B{[}1{]})\textsuperscript{2+(y-B{[}2{]})}2)

\textgreater fcontour(``d2'',xmin=-2,xmax=2,ymin=-3,ymax=1,hue=1):

\begin{figure}
\centering
\pandocbounded{\includegraphics[keepaspectratio]{images/Adiyatma_23030630062__EMT4Geometry-107.png}}
\caption{images/Adiyatma\_23030630062\_\_EMT4Geometry-107.png}
\end{figure}

Grafiknya lebih menarik:

\textgreater plot3d(``d2'',xmin=-2,xmax=2,ymin=-3,ymax=1):

\begin{figure}
\centering
\pandocbounded{\includegraphics[keepaspectratio]{images/Adiyatma_23030630062__EMT4Geometry-108.png}}
\caption{images/Adiyatma\_23030630062\_\_EMT4Geometry-108.png}
\end{figure}

Pembatasan garis (AB) lebih terkenal:

\textgreater plot2d(``abs(x+1)+abs(x-1)'',xmin=-3,xmax=3):

\begin{figure}
\centering
\pandocbounded{\includegraphics[keepaspectratio]{images/Adiyatma_23030630062__EMT4Geometry-109.png}}
\caption{images/Adiyatma\_23030630062\_\_EMT4Geometry-109.png}
\end{figure}

\section{Poin Ketiga}\label{poin-ketiga}

Sekarang hal-hal yang kurang sederhana: Ini sedikit kurang terkenal bahwa MA+MB+MC mencapai minimum pada satu titik pesawat tetapi untuk menentukan itu kurang sederhana:

\begin{enumerate}
\def\labelenumi{\arabic{enumi})}
\tightlist
\item
  Jika salah satu sudut segitiga ABC lebih dari 120° (katakanlah di A), maka minimum dicapai pada titik ini (misalnya AB+AC).
\end{enumerate}

Contoh:

\textgreater C={[}-4,1{]};

\textgreater function d3(x,y):=d2(x,y)+sqrt((x-C{[}1{]})\textsuperscript{2+(y-C{[}2{]})}2)

\textgreater plot3d(``d3'',xmin=-5,xmax=3,ymin=-4,ymax=4);

\textgreater insimg;

\begin{figure}
\centering
\pandocbounded{\includegraphics[keepaspectratio]{images/Adiyatma_23030630062__EMT4Geometry-110.png}}
\caption{images/Adiyatma\_23030630062\_\_EMT4Geometry-110.png}
\end{figure}

\textgreater fcontour(``d3'',xmin=-4,xmax=1,ymin=-2,ymax=2,hue=1,title=``The minimum is on A'');

\textgreater P=(A\_B\_C\_A)'; plot2d(P{[}1{]},P{[}2{]},add=1,color=12);

\textgreater insimg;

\begin{figure}
\centering
\pandocbounded{\includegraphics[keepaspectratio]{images/Adiyatma_23030630062__EMT4Geometry-111.png}}
\caption{images/Adiyatma\_23030630062\_\_EMT4Geometry-111.png}
\end{figure}

\begin{enumerate}
\def\labelenumi{\arabic{enumi})}
\setcounter{enumi}{1}
\tightlist
\item
  Tetapi jika semua sudut segitiga ABC kurang dari 120 °, minimumnya adalah pada titik F di bagian dalam segitiga, yang merupakan satu-satunya titik yang melihat sisi-sisi ABC dengan sudut yang sama (maka masing-masing 120 ° ):
\end{enumerate}

\textgreater C={[}-0.5,1{]};

\textgreater plot3d(``d3'',xmin=-2,xmax=2,ymin=-2,ymax=2):

\begin{figure}
\centering
\pandocbounded{\includegraphics[keepaspectratio]{images/Adiyatma_23030630062__EMT4Geometry-112.png}}
\caption{images/Adiyatma\_23030630062\_\_EMT4Geometry-112.png}
\end{figure}

\textgreater fcontour(``d3'',xmin=-2,xmax=2,ymin=-2,ymax=2,hue=1,title=``The Fermat point'');

\textgreater P=(A\_B\_C\_A)'; plot2d(P{[}1{]},P{[}2{]},add=1,color=12);

\textgreater insimg;

\begin{figure}
\centering
\pandocbounded{\includegraphics[keepaspectratio]{images/Adiyatma_23030630062__EMT4Geometry-113.png}}
\caption{images/Adiyatma\_23030630062\_\_EMT4Geometry-113.png}
\end{figure}

Merupakan kegiatan yang menarik untuk mewujudkan gambar di atas dengan perangkat lunak geometri; misalnya, saya tahu soft yang ditulis di Jawa yang memiliki instruksi ``garis kontur'' \ldots{}

Semua ini di atas telah ditemukan oleh seorang hakim Perancis bernama Pierre de Fermat; dia menulis surat kepada dilettants lain seperti pendeta Marin Mersenne dan Blaise Pascal yang bekerja di pajak penghasilan. Jadi titik unik F sedemikian rupa sehingga FA+FB+FC minimal, disebut titik Fermat segitiga. Tetapi tampaknya beberapa tahun sebelumnya, Torriccelli Italia telah menemukan titik ini sebelum Fermat melakukannya! Bagaimanapun tradisinya adalah mencatat poin ini F\ldots{}

\section{Poin Empat}\label{poin-empat}

Langkah selanjutnya adalah menambahkan 4 titik D dan mencoba meminimalkan MA+MB+MC+MD; katakan bahwa Anda adalah operator TV kabel dan ingin mencari di bidang mana Anda harus meletakkan antena sehingga Anda dapat memberi makan empat desa dan menggunakan panjang kabel sesedikit mungkin!

\textgreater D={[}1,1{]};

\textgreater function d4(x,y):=d3(x,y)+sqrt((x-D{[}1{]})\textsuperscript{2+(y-D{[}2{]})}2)

\textgreater plot3d(``d4'',xmin=-1.5,xmax=1.5,ymin=-1.5,ymax=1.5):

\begin{figure}
\centering
\pandocbounded{\includegraphics[keepaspectratio]{images/Adiyatma_23030630062__EMT4Geometry-114.png}}
\caption{images/Adiyatma\_23030630062\_\_EMT4Geometry-114.png}
\end{figure}

\textgreater fcontour(``d4'',xmin=-1.5,xmax=1.5,ymin=-1.5,ymax=1.5,hue=1);

\textgreater P=(A\_B\_C\_D)'; plot2d(P{[}1{]},P{[}2{]},points=1,add=1,color=12);

\textgreater insimg;

\begin{figure}
\centering
\pandocbounded{\includegraphics[keepaspectratio]{images/Adiyatma_23030630062__EMT4Geometry-115.png}}
\caption{images/Adiyatma\_23030630062\_\_EMT4Geometry-115.png}
\end{figure}

Masih ada minimum dan tidak tercapai di salah satu simpul A, B, C atau D:

\textgreater function f(x):=d4(x{[}1{]},x{[}2{]})

\textgreater neldermin(``f'',{[}0.2,0.2{]})

\begin{verbatim}
[0.142858,  0.142857]
\end{verbatim}

Tampaknya dalam kasus ini, koordinat titik optimal adalah rasional atau mendekati rasional\ldots{}

Sekarang ABCD adalah persegi, kami berharap bahwa titik optimal akan menjadi pusat ABCD:

\textgreater C={[}-1,1{]};

\textgreater plot3d(``d4'',xmin=-1,xmax=1,ymin=-1,ymax=1):

\begin{figure}
\centering
\pandocbounded{\includegraphics[keepaspectratio]{images/Adiyatma_23030630062__EMT4Geometry-116.png}}
\caption{images/Adiyatma\_23030630062\_\_EMT4Geometry-116.png}
\end{figure}

\textgreater fcontour(``d4'',xmin=-1.5,xmax=1.5,ymin=-1.5,ymax=1.5,hue=1);

\textgreater P=(A\_B\_C\_D)'; plot2d(P{[}1{]},P{[}2{]},add=1,color=12,points=1);

\textgreater insimg;

\begin{figure}
\centering
\pandocbounded{\includegraphics[keepaspectratio]{images/Adiyatma_23030630062__EMT4Geometry-117.png}}
\caption{images/Adiyatma\_23030630062\_\_EMT4Geometry-117.png}
\end{figure}

\section{Contoh 7: Bola Dandelin dengan Povray}\label{contoh-7-bola-dandelin-dengan-povray}

Anda dapat menjalankan demonstrasi ini, jika Anda telah menginstal Povray, dan pvengine.exe di jalur program.

Pertama kita hitung jari-jari bola.

Jika Anda melihat gambar di bawah, Anda melihat bahwa kita membutuhkan dua lingkaran yang menyentuh dua garis yang membentuk kerucut, dan satu garis yang membentuk bidang yang memotong kerucut.

Kami menggunakan file geometri.e dari Euler untuk ini.

\textgreater load geometry;

Pertama dua garis yang membentuk kerucut.

\textgreater g1 \&= lineThrough({[}0,0{]},{[}1,a{]})

\begin{verbatim}
                             [- a, 1, 0]
\end{verbatim}

\textgreater g2 \&= lineThrough({[}0,0{]},{[}-1,a{]})

\begin{verbatim}
                            [- a, - 1, 0]
\end{verbatim}

Kemudian baris ketiga.

\textgreater g \&= lineThrough({[}-1,0{]},{[}1,1{]})

\begin{verbatim}
                             [- 1, 2, 1]
\end{verbatim}

Kami merencanakan semuanya sejauh ini.

\textgreater setPlotRange(-1,1,0,2);

\textgreater color(black); plotLine(g(),``\,``)

\textgreater a:=2; color(blue); plotLine(g1(),``\,``), plotLine(g2(),''\,``):

\begin{figure}
\centering
\pandocbounded{\includegraphics[keepaspectratio]{images/Adiyatma_23030630062__EMT4Geometry-118.png}}
\caption{images/Adiyatma\_23030630062\_\_EMT4Geometry-118.png}
\end{figure}

Sekarang kita ambil titik umum pada sumbu y.

\textgreater P \&= {[}0,u{]}

\begin{verbatim}
                                [0, u]
\end{verbatim}

Hitung jarak ke g1.

\textgreater d1 \&= distance(P,projectToLine(P,g1)); \$d1

\[\sqrt{\left(\frac{a^2\,u}{a^2+1}-u\right)^2+\frac{a^2\,u^2}{\left(a ^2+1\right)^2}}\]Hitung jarak ke g.

\textgreater d \&= distance(P,projectToLine(P,g)); \$d

\[\sqrt{\left(\frac{u+2}{5}-u\right)^2+\frac{\left(2\,u-1\right)^2}{ 25}}\]Dan temukan pusat kedua lingkaran yang jaraknya sama.

\textgreater sol \&= solve(d1\textsuperscript{2=d}2,u); \$sol

\[\left[ u=\frac{-\sqrt{5}\,\sqrt{a^2+1}+2\,a^2+2}{4\,a^2-1} , u= \frac{\sqrt{5}\,\sqrt{a^2+1}+2\,a^2+2}{4\,a^2-1} \right]\]Ada dua solusi.

Kami mengevaluasi solusi simbolis, dan menemukan kedua pusat, dan kedua jarak.

\textgreater u := sol()

\begin{verbatim}
[0.333333,  1]
\end{verbatim}

\textgreater dd := d()

\begin{verbatim}
[0.149071,  0.447214]
\end{verbatim}

Plot lingkaran ke dalam gambar.

\textgreater color(red);

\textgreater plotCircle(circleWithCenter({[}0,u{[}1{]}{]},dd{[}1{]}),``\,``);

\textgreater plotCircle(circleWithCenter({[}0,u{[}2{]}{]},dd{[}2{]}),``\,``);

\textgreater insimg;

\begin{figure}
\centering
\pandocbounded{\includegraphics[keepaspectratio]{images/Adiyatma_23030630062__EMT4Geometry-122.png}}
\caption{images/Adiyatma\_23030630062\_\_EMT4Geometry-122.png}
\end{figure}

\section{Plot dengan Povray}\label{plot-dengan-povray}

Selanjutnya kami merencanakan semuanya dengan Povray. Perhatikan bahwa Anda mengubah perintah apa pun dalam urutan perintah Povray berikut, dan menjalankan kembali semua perintah dengan Shift-Return.

Pertama kita memuat fungsi povray.

\textgreater load povray;

\textgreater defaultpovray=``C:\textbackslash Program Files\textbackslash POV-Ray\textbackslash v3.7\textbackslash bin\textbackslash pvengine.exe''

\begin{verbatim}
C:\Program Files\POV-Ray\v3.7\bin\pvengine.exe
\end{verbatim}

Kami mengatur adegan dengan tepat.

\textgreater povstart(zoom=11,center={[}0,0,0.5{]},height=10°,angle=140°);

Selanjutnya kita menulis dua bidang ke file Povray.

\textgreater writeln(povsphere({[}0,0,u{[}1{]}{]},dd{[}1{]},povlook(red)));

\textgreater writeln(povsphere({[}0,0,u{[}2{]}{]},dd{[}2{]},povlook(red)));

Dan kerucutnya, transparan.

\textgreater writeln(povcone({[}0,0,0{]},0,{[}0,0,a{]},1,povlook(lightgray,1)));

Kami menghasilkan bidang terbatas pada kerucut.

\textgreater gp=g();

\textgreater pc=povcone({[}0,0,0{]},0,{[}0,0,a{]},1,``\,``);

\textgreater vp={[}gp{[}1{]},0,gp{[}2{]}{]}; dp=gp{[}3{]};

\textgreater writeln(povplane(vp,dp,povlook(blue,0.5),pc));

Sekarang kita menghasilkan dua titik pada lingkaran, di mana bola menyentuh kerucut.

\textgreater function turnz(v) := return

\textgreater P1=projectToLine({[}0,u{[}1{]}{]},g1()); P1=turnz({[}P1{[}1{]},0,P1{[}2{]}{]});

\textgreater writeln(povpoint(P1,povlook(yellow)));

\textgreater P2=projectToLine({[}0,u{[}2{]}{]},g1()); P2=turnz({[}P2{[}1{]},0,P2{[}2{]}{]});

\textgreater writeln(povpoint(P2,povlook(yellow)));

Kemudian kami menghasilkan dua titik di mana bola menyentuh bidang. Ini adalah fokus dari elips.

\textgreater P3=projectToLine({[}0,u{[}1{]}{]},g()); P3={[}P3{[}1{]},0,P3{[}2{]}{]};

\textgreater writeln(povpoint(P3,povlook(yellow)));

\textgreater P4=projectToLine({[}0,u{[}2{]}{]},g()); P4={[}P4{[}1{]},0,P4{[}2{]}{]};

\textgreater writeln(povpoint(P4,povlook(yellow)));

Selanjutnya kita hitung perpotongan P1P2 dengan bidang.

\textgreater t1=scalp(vp,P1)-dp; t2=scalp(vp,P2)-dp; P5=P1+t1/(t1-t2)*(P2-P1);

\textgreater writeln(povpoint(P5,povlook(yellow)));

Kami menghubungkan titik-titik dengan segmen garis.

\textgreater writeln(povsegment(P1,P2,povlook(yellow)));

\textgreater writeln(povsegment(P5,P3,povlook(yellow)));

\textgreater writeln(povsegment(P5,P4,povlook(yellow)));

Sekarang kita menghasilkan pita abu-abu, di mana bola menyentuh kerucut.

\textgreater pcw=povcone({[}0,0,0{]},0,{[}0,0,a{]},1.01);

\textgreater pc1=povcylinder({[}0,0,P1{[}3{]}-defaultpointsize/2{]},{[}0,0,P1{[}3{]}+defaultpointsize/2{]},1);

\textgreater writeln(povintersection({[}pcw,pc1{]},povlook(gray)));

\textgreater pc2=povcylinder({[}0,0,P2{[}3{]}-defaultpointsize/2{]},{[}0,0,P2{[}3{]}+defaultpointsize/2{]},1);

\textgreater writeln(povintersection({[}pcw,pc2{]},povlook(gray)));

Mulai program Povray.

\textgreater povend();

\begin{figure}
\centering
\pandocbounded{\includegraphics[keepaspectratio]{images/Adiyatma_23030630062__EMT4Geometry-123.png}}
\caption{images/Adiyatma\_23030630062\_\_EMT4Geometry-123.png}
\end{figure}

Untuk mendapatkan Anaglyph ini kita perlu memasukkan semuanya ke dalam fungsi scene. Fungsi ini akan digunakan dua kali kemudian.

\textgreater function scene () \ldots{}

\begin{verbatim}
global a,u,dd,g,g1,defaultpointsize;
writeln(povsphere([0,0,u[1]],dd[1],povlook(red)));
writeln(povsphere([0,0,u[2]],dd[2],povlook(red)));
writeln(povcone([0,0,0],0,[0,0,a],1,povlook(lightgray,1)));
gp=g();
pc=povcone([0,0,0],0,[0,0,a],1,"");
vp=[gp[1],0,gp[2]]; dp=gp[3];
writeln(povplane(vp,dp,povlook(blue,0.5),pc));
P1=projectToLine([0,u[1]],g1()); P1=turnz([P1[1],0,P1[2]]);
writeln(povpoint(P1,povlook(yellow)));
P2=projectToLine([0,u[2]],g1()); P2=turnz([P2[1],0,P2[2]]);
writeln(povpoint(P2,povlook(yellow)));
P3=projectToLine([0,u[1]],g()); P3=[P3[1],0,P3[2]];
writeln(povpoint(P3,povlook(yellow)));
P4=projectToLine([0,u[2]],g()); P4=[P4[1],0,P4[2]];
writeln(povpoint(P4,povlook(yellow)));
t1=scalp(vp,P1)-dp; t2=scalp(vp,P2)-dp; P5=P1+t1/(t1-t2)*(P2-P1);
writeln(povpoint(P5,povlook(yellow)));
writeln(povsegment(P1,P2,povlook(yellow)));
writeln(povsegment(P5,P3,povlook(yellow)));
writeln(povsegment(P5,P4,povlook(yellow)));
pcw=povcone([0,0,0],0,[0,0,a],1.01);
pc1=povcylinder([0,0,P1[3]-defaultpointsize/2],[0,0,P1[3]+defaultpointsize/2],1);
writeln(povintersection([pcw,pc1],povlook(gray)));
pc2=povcylinder([0,0,P2[3]-defaultpointsize/2],[0,0,P2[3]+defaultpointsize/2],1);
writeln(povintersection([pcw,pc2],povlook(gray)));
endfunction
\end{verbatim}

Anda membutuhkan kacamata merah/sian untuk menghargai efek berikut.

\textgreater povanaglyph(``scene'',zoom=11,center={[}0,0,0.5{]},height=10°,angle=140°);

\begin{figure}
\centering
\pandocbounded{\includegraphics[keepaspectratio]{images/Adiyatma_23030630062__EMT4Geometry-124.png}}
\caption{images/Adiyatma\_23030630062\_\_EMT4Geometry-124.png}
\end{figure}

\chapter{Contoh 8: Geometri Bumi}\label{contoh-8-geometri-bumi}

Dalam buku catatan ini, kami ingin melakukan beberapa perhitungan sferis. Fungsi-fungsi tersebut terdapat dalam file ``spherical.e'' di folder contoh. Kita perlu memuat file itu terlebih dahulu.

\textgreater load ``spherical.e'';

Untuk memasukkan posisi geografis, kami menggunakan vektor dengan dua koordinat dalam radian (utara dan timur, nilai negatif untuk selatan dan barat). Berikut koordinat Kampus FMIPA UNY.

\textgreater FMIPA={[}rad(-7,-46.467),rad(110,23.05){]}

\begin{verbatim}
[-0.13569,  1.92657]
\end{verbatim}

Anda dapat mencetak posisi ini dengan sposprint (cetak posisi spherical).

\textgreater sposprint(FMIPA) // posisi garis lintang dan garis bujur FMIPA UNY

\begin{verbatim}
S 7°46.467' E 110°23.050'
\end{verbatim}

Mari kita tambahkan dua kota lagi, Solo dan Semarang.

\textgreater Solo={[}rad(-7,-34.333),rad(110,49.683){]}; Semarang={[}rad(-6,-59.05),rad(110,24.533){]};

\textgreater sposprint(Solo), sposprint(Semarang),

\begin{verbatim}
S 7°34.333' E 110°49.683'
S 6°59.050' E 110°24.533'
\end{verbatim}

Pertama kita menghitung vektor dari satu ke yang lain pada bola ideal. Vektor ini {[}pos,jarak{]} dalam radian. Untuk menghitung jarak di bumi, kita kalikan dengan jari-jari bumi pada garis lintang 7°.

\textgreater br=svector(FMIPA,Solo); degprint(br{[}1{]}), br{[}2{]}*rearth(7°)-\textgreater km // perkiraan jarak FMIPA-Solo

\begin{verbatim}
65°20'26.60''
53.8945384608
\end{verbatim}

Ini adalah perkiraan yang baik. Rutinitas berikut menggunakan perkiraan yang lebih baik. Pada jarak yang begitu pendek hasilnya hampir sama.

\textgreater esdist(FMIPA,Semarang)-\textgreater'' km'', // perkiraan jarak FMIPA-Semarang

\begin{verbatim}
88.0114026318 km
\end{verbatim}

Ada fungsi untuk heading, dengan mempertimbangkan bentuk elips bumi. Sekali lagi, kami mencetak dengan cara yang canggih.

\textgreater sdegprint(esdir(FMIPA,Solo))

\begin{verbatim}
     65.34°
\end{verbatim}

Sudut segitiga melebihi 180° pada bola.

\textgreater asum=sangle(Solo,FMIPA,Semarang)+sangle(FMIPA,Solo,Semarang)+sangle(FMIPA,Semarang,Solo); degprint(asum)

\begin{verbatim}
180°0'10.77''
\end{verbatim}

Ini dapat digunakan untuk menghitung luas segitiga. Catatan: Untuk segitiga kecil, ini tidak akurat karena kesalahan pengurangan dalam asum-pi.

\textgreater(asum-pi)*rearth(48°)\^{}2-\textgreater'' km\^{}2'', // perkiraan luas segitiga FMIPA-Solo-Semarang,

\begin{verbatim}
2116.02948749 km^2
\end{verbatim}

Ada fungsi untuk ini, yang menggunakan garis lintang rata-rata segitiga untuk menghitung jari-jari bumi, dan menangani kesalahan pembulatan untuk segitiga yang sangat kecil.

\textgreater esarea(Solo,FMIPA,Semarang)-\textgreater'' km\^{}2'', //perkiraan yang sama dengan fungsi esarea()

\begin{verbatim}
2123.64310526 km^2
\end{verbatim}

Kita juga dapat menambahkan vektor ke posisi. Sebuah vektor berisi heading dan jarak, keduanya dalam radian. Untuk mendapatkan vektor, kami menggunakan vektor. Untuk menambahkan vektor ke posisi, kami menggunakan vektor sadd.

\textgreater v=svector(FMIPA,Solo); sposprint(saddvector(FMIPA,v)), sposprint(Solo),

\begin{verbatim}
S 7°34.333' E 110°49.683'
S 7°34.333' E 110°49.683'
\end{verbatim}

Fungsi-fungsi ini mengasumsikan bola yang ideal. Hal yang sama di bumi.

\textgreater sposprint(esadd(FMIPA,esdir(FMIPA,Solo),esdist(FMIPA,Solo))), sposprint(Solo),

\begin{verbatim}
S 7°34.333' E 110°49.683'
S 7°34.333' E 110°49.683'
\end{verbatim}

Mari kita beralih ke contoh yang lebih besar, Tugu Jogja dan Monas Jakarta (menggunakan Google Earth untuk mencari koordinatnya).

\textgreater Tugu={[}-7.7833°,110.3661°{]}; Monas={[}-6.175°,106.811944°{]};

\textgreater sposprint(Tugu), sposprint(Monas)

\begin{verbatim}
S 7°46.998' E 110°21.966'
S 6°10.500' E 106°48.717'
\end{verbatim}

Menurut Google Earth, jaraknya adalah 429,66 km. Kami mendapatkan pendekatan yang baik.

\textgreater esdist(Tugu,Monas)-\textgreater'' km'', // perkiraan jarak Tugu Jogja - Monas Jakarta

\begin{verbatim}
431.565659488 km
\end{verbatim}

Judulnya sama dengan judul yang dihitung di Google Earth.

\textgreater degprint(esdir(Tugu,Monas))

\begin{verbatim}
294°17'2.85''
\end{verbatim}

Namun, kita tidak lagi mendapatkan posisi target yang tepat, jika kita menambahkan heading dan jarak ke posisi semula. Hal ini terjadi, karena kita tidak menghitung fungsi invers secara tepat, tetapi mengambil perkiraan jari-jari bumi di sepanjang jalan.

\textgreater sposprint(esadd(Tugu,esdir(Tugu,Monas),esdist(Tugu,Monas)))

\begin{verbatim}
S 6°10.500' E 106°48.717'
\end{verbatim}

Namun, kesalahannya tidak besar.

\textgreater sposprint(Monas),

\begin{verbatim}
S 6°10.500' E 106°48.717'
\end{verbatim}

Tentu kita tidak bisa berlayar dengan tujuan yang sama dari satu tujuan ke tujuan lainnya, jika kita ingin menempuh jalur terpendek. Bayangkan, Anda terbang NE mulai dari titik mana pun di bumi. Kemudian Anda akan berputar ke kutub utara. Lingkaran besar tidak mengikuti heading yang konstan!

Perhitungan berikut menunjukkan bahwa kami jauh dari tujuan yang benar, jika kami menggunakan pos yang sama selama perjalanan kami.

\textgreater dist=esdist(Tugu,Monas); hd=esdir(Tugu,Monas);

Sekarang kita tambahkan 10 kali sepersepuluh dari jarak, menggunakan pos ke Monas, kita sampai di Tugu.

\textgreater p=Tugu; loop 1 to 10; p=esadd(p,hd,dist/10); end;

Hasilnya jauh.

\textgreater sposprint(p), skmprint(esdist(p,Monas))

\begin{verbatim}
S 6°11.250' E 106°48.372'
     1.529km
\end{verbatim}

Sebagai contoh lain, mari kita ambil dua titik di bumi pada garis lintang yang sama.

\textgreater P1={[}30°,10°{]}; P2={[}30°,50°{]};

Jalur terpendek dari P1 ke P2 bukanlah lingkaran garis lintang 30°, melainkan jalur terpendek yang dimulai 10° lebih jauh ke utara di P1.

\textgreater sdegprint(esdir(P1,P2))

\begin{verbatim}
     79.69°
\end{verbatim}

Tapi, jika kita mengikuti pembacaan kompas ini, kita akan berputar ke kutub utara! Jadi kita harus menyesuaikan arah kita di sepanjang jalan. Untuk tujuan kasar, kami menyesuaikannya pada 1/10 dari total jarak.

\textgreater p=P1; dist=esdist(P1,P2); \ldots{}\\
\textgreater{} loop 1 to 10; dir=esdir(p,P2); sdegprint(dir), p=esadd(p,dir,dist/10); end;

\begin{verbatim}
     79.69°
     81.67°
     83.71°
     85.78°
     87.89°
     90.00°
     92.12°
     94.22°
     96.29°
     98.33°
\end{verbatim}

Jaraknya tidak tepat, karena kita akan menambahkan sedikit kesalahan, jika kita mengikuti heading yang sama terlalu lama.

\textgreater skmprint(esdist(p,P2))

\begin{verbatim}
     0.203km
\end{verbatim}

Kami mendapatkan perkiraan yang baik, jika kami menyesuaikan pos setelah setiap 1/100 dari total jarak dari Tugu ke Monas.

\textgreater p=Tugu; dist=esdist(Tugu,Monas); \ldots{}\\
\textgreater{} loop 1 to 100; p=esadd(p,esdir(p,Monas),dist/100); end;

\textgreater skmprint(esdist(p,Monas))

\begin{verbatim}
     0.000km
\end{verbatim}

Untuk keperluan navigasi, kita bisa mendapatkan urutan posisi GPS di sepanjang lingkaran besar menuju Monas dengan fungsi navigasi.

\textgreater load spherical; v=navigate(Tugu,Monas,10); \ldots{}\\
\textgreater{} loop 1 to rows(v); sposprint(v{[}\#{]}), end;

\begin{verbatim}
S 7°46.998' E 110°21.966'
S 7°37.422' E 110°0.573'
S 7°27.829' E 109°39.196'
S 7°18.219' E 109°17.834'
S 7°8.592' E 108°56.488'
S 6°58.948' E 108°35.157'
S 6°49.289' E 108°13.841'
S 6°39.614' E 107°52.539'
S 6°29.924' E 107°31.251'
S 6°20.219' E 107°9.977'
S 6°10.500' E 106°48.717'
\end{verbatim}

Kami menulis sebuah fungsi, yang memplot bumi, dua posisi, dan posisi di antaranya.

\textgreater function testplot \ldots{}

\begin{verbatim}
useglobal;
plotearth;
plotpos(Tugu,"Tugu Jogja"); plotpos(Monas,"Tugu Monas");
plotposline(v);
endfunction
\end{verbatim}

Sekarang rencanakan semuanya.

\textgreater plot3d(``testplot'',angle=25, height=6,\textgreater own,\textgreater user,zoom=4):

\begin{figure}
\centering
\pandocbounded{\includegraphics[keepaspectratio]{images/Adiyatma_23030630062__EMT4Geometry-125.png}}
\caption{images/Adiyatma\_23030630062\_\_EMT4Geometry-125.png}
\end{figure}

Atau gunakan plot3d untuk mendapatkan tampilan anaglyph. Ini terlihat sangat bagus dengan kacamata merah/sian.

\textgreater plot3d(``testplot'',angle=25,height=6,distance=5,own=1,anaglyph=1,zoom=4):

\begin{figure}
\centering
\pandocbounded{\includegraphics[keepaspectratio]{images/Adiyatma_23030630062__EMT4Geometry-126.png}}
\caption{images/Adiyatma\_23030630062\_\_EMT4Geometry-126.png}
\end{figure}

\chapter{Latihan}\label{latihan-1}

\begin{enumerate}
\def\labelenumi{\arabic{enumi}.}
\tightlist
\item
  Gambarlah segi-n beraturan jika diketahui titik pusat O, n, dan jarak titik pusat ke titik-titik sudut segi-n tersebut (jari-jari lingkaran luar segi-n), r.
\end{enumerate}

Petunjuk:

\begin{itemize}
\item
  Besar sudut pusat yang menghadap masing-masing sisi segi-n adalah
\item
  (360/n).
\item
  Titik-titik sudut segi-n merupakan perpotongan lingkaran luar segi-n
\item
  dan garis-garis yang melalui pusat dan saling membentuk sudut sebesar
\item
  kelipatan (360/n).
\item
  Untuk n ganjil, pilih salah satu titik sudut adalah di atas.
\item
  Untuk n genap, pilih 2 titik di kanan dan kiri lurus dengan titik
\item
  pusat.
\item
  Anda dapat menggambar segi-3, 4, 5, 6, 7, dst beraturan.
\end{itemize}

\textgreater load geometry

\begin{verbatim}
Numerical and symbolic geometry.
\end{verbatim}

\textgreater setPlotRange(-3.5,3.5,-3.5,3.5);

\textgreater A={[}-2,-2{]}; plotPoint(A,``A'');

\textgreater B={[}2,-2{]}; plotPoint(B,``B'');

\textgreater C={[}0,3{]}; plotPoint(C,``C'');

\textgreater plotSegment(A,B,``c'');

\textgreater plotSegment(B,C,``a'');

\textgreater plotSegment(A,C,``b'');

\textgreater aspect(1):

\begin{figure}
\centering
\pandocbounded{\includegraphics[keepaspectratio]{images/Adiyatma_23030630062__EMT4Geometry-127.png}}
\caption{images/Adiyatma\_23030630062\_\_EMT4Geometry-127.png}
\end{figure}

\textgreater c=circleThrough(A,B,C);

\textgreater R=getCircleRadius(c);

\textgreater O=getCircleCenter(c);

\textgreater plotPoint(O,``O'');

\textgreater l=angleBisector(A,C,B);

\textgreater color(2); plotLine(l); color(1);

\textgreater plotCircle(c,``Lingkaran luar segitiga ABC''):

\begin{figure}
\centering
\pandocbounded{\includegraphics[keepaspectratio]{images/Adiyatma_23030630062__EMT4Geometry-128.png}}
\caption{images/Adiyatma\_23030630062\_\_EMT4Geometry-128.png}
\end{figure}

\begin{enumerate}
\def\labelenumi{\arabic{enumi}.}
\setcounter{enumi}{1}
\tightlist
\item
  Gambarlah suatu parabola yang melalui 3 titik yang diketahui.
\end{enumerate}

Petunjuk:

\begin{itemize}
\item
  Misalkan persamaan parabolanya y= ax\^{}2+bx+c.
\item
  Substitusikan koordinat titik-titik yang diketahui ke persamaan tersebut.
\item
  Selesaikan SPL yang terbentuk untuk mendapatkan nilai-nilai a, b, c.
\end{itemize}

\textgreater load geometry;

\textgreater setPlotRange(5); P={[}2,0{]}; Q={[}4,0{]}; R={[}0,-4{]};

\textgreater plotPoint(P,``P''); plotPoint(Q,``Q''); plotPoint(R,``R''):

\begin{figure}
\centering
\pandocbounded{\includegraphics[keepaspectratio]{images/Adiyatma_23030630062__EMT4Geometry-129.png}}
\caption{images/Adiyatma\_23030630062\_\_EMT4Geometry-129.png}
\end{figure}

\textgreater sol \&= solve({[}a+b=-c,16*a+4*b=-c,c=-4{]},{[}a,b,c{]})

\begin{verbatim}
                     [[a = - 1, b = 5, c = - 4]]
\end{verbatim}

\textgreater function y\&=-x\^{}2+5*x-4

\begin{verbatim}
                               2
                            - x  + 5 x - 4
\end{verbatim}

\textgreater plot2d(``-x\^{}2+5*x-4'',-5,5,-5,5):

\begin{figure}
\centering
\pandocbounded{\includegraphics[keepaspectratio]{images/Adiyatma_23030630062__EMT4Geometry-130.png}}
\caption{images/Adiyatma\_23030630062\_\_EMT4Geometry-130.png}
\end{figure}

\begin{enumerate}
\def\labelenumi{\arabic{enumi}.}
\setcounter{enumi}{2}
\item
  Gambarlah suatu segi-4 yang diketahui keempat titik sudutnya, misalnya A, B, C, D.

  \begin{itemize}
  \item
    Tentukan apakah segi-4 tersebut merupakan segi-4 garis singgung (sisinya-sisintya merupakan garis singgung lingkaran yang sama yakni lingkaran dalam segi-4 tersebut).
  \item
    Suatu segi-4 merupakan segi-4 garis singgung apabila keempat garis bagi sudutnya bertemu di satu titik.
  \item
    Jika segi-4 tersebut merupakan segi-4 garis singgung, gambar lingkaran dalamnya.
  \item
    Tunjukkan bahwa syarat suatu segi-4 merupakan segi-4 garis singgung apabila hasil kali panjang sisi-sisi yang berhadapan sama.
  \end{itemize}
\end{enumerate}

\textgreater load geometry

\begin{verbatim}
Numerical and symbolic geometry.
\end{verbatim}

\textgreater setPlotRange(-4.5,4.5,-4.5,4.5);

\textgreater A={[}-3,-3{]}; plotPoint(A,``A'');

\textgreater B={[}3,-3{]}; plotPoint(B,``B'');

\textgreater C={[}3,3{]}; plotPoint(C,``C'');

\textgreater D={[}-3,3{]}; plotPoint(D,``D'');

\textgreater plotSegment(A,B,``\,``);

\textgreater plotSegment(B,C,``\,``);

\textgreater plotSegment(C,D,``\,``);

\textgreater plotSegment(A,D,``\,``);

\textgreater aspect(1):

\begin{figure}
\centering
\pandocbounded{\includegraphics[keepaspectratio]{images/Adiyatma_23030630062__EMT4Geometry-131.png}}
\caption{images/Adiyatma\_23030630062\_\_EMT4Geometry-131.png}
\end{figure}

\textgreater l=angleBisector(A,B,C);

\textgreater m=angleBisector(B,C,D);

\textgreater P=lineIntersection(l,m);

\textgreater color(5); plotLine(l); plotLine(m); color(1);

\textgreater plotPoint(P,``P''):

\begin{figure}
\centering
\pandocbounded{\includegraphics[keepaspectratio]{images/Adiyatma_23030630062__EMT4Geometry-132.png}}
\caption{images/Adiyatma\_23030630062\_\_EMT4Geometry-132.png}
\end{figure}

Dari gambar diatas terlihat bahwa keempat garis bagi sudutnya bertemu di satu titik yaitu titik P.

\textgreater r=norm(P-projectToLine(P,lineThrough(A,B)));

\textgreater plotCircle(circleWithCenter(P,r),``Lingkaran dalam segiempat ABCD''):

\begin{figure}
\centering
\pandocbounded{\includegraphics[keepaspectratio]{images/Adiyatma_23030630062__EMT4Geometry-133.png}}
\caption{images/Adiyatma\_23030630062\_\_EMT4Geometry-133.png}
\end{figure}

Dari gambar diatas, terlihat bahwa sisi-sisinya merupakan garis singgung lingkaran yang sama yaitu lingkaran dalam segiempat.

Akan ditunjukkan bahwa hasil kali panjang sisi-sisi yang berhadapan sama.

\textgreater AB=norm(A-B) //panjang sisi AB

\begin{verbatim}
6
\end{verbatim}

\textgreater CD=norm(C-D) //panjang sisi CD

\begin{verbatim}
6
\end{verbatim}

\textgreater AD=norm(A-D) //panjang sisi AD

\begin{verbatim}
6
\end{verbatim}

\textgreater BC=norm(B-C) //panjang sisi BC

\begin{verbatim}
6
\end{verbatim}

\textgreater AB.CD

\begin{verbatim}
36
\end{verbatim}

\textgreater AD.BC

\begin{verbatim}
36
\end{verbatim}

Terbukti bahwa hasil kali panjang sisi-sisi yang berhadapan sama yaitu 36. Jadi dapat dipastikan bahwa segiempat tersebut merupakan segiempat garis singgung.

\begin{enumerate}
\def\labelenumi{\arabic{enumi}.}
\setcounter{enumi}{3}
\tightlist
\item
  Gambarlah suatu ellips jika diketahui kedua titik fokusnya, misalnya P dan Q. Ingat ellips dengan fokus P dan Q adalah tempat kedudukan titik-titik yang jumlah jarak ke P dan ke Q selalu sama (konstan).
\end{enumerate}

Penyelesaian :

Diketahui kedua titik fokus P = {[}-1,-1{]} dan Q = {[}1,-1{]}

\textgreater P={[}-1,-1{]}; Q={[}1,-1{]};

\textgreater function d1(x,y):=sqrt((x-P{[}1{]})\textsuperscript{2+(y-P{[}2{]})}2)

\textgreater Q={[}1,-1{]}; function d2(x,y):=sqrt((x-P{[}1{]})\textsuperscript{2+(y-P{[}2{]})}2)+sqrt((x-Q{[}1{]})\textsuperscript{2+(y-Q{[}2{]})}2)

\textgreater fcontour(``d2'',xmin=-2,xmax=2,ymin=-3,ymax=1,hue=1):

\begin{figure}
\centering
\pandocbounded{\includegraphics[keepaspectratio]{images/Adiyatma_23030630062__EMT4Geometry-134.png}}
\caption{images/Adiyatma\_23030630062\_\_EMT4Geometry-134.png}
\end{figure}

Grafik yang lebih menarik

\textgreater plot3d(``d2'',xmin=-2,xmax=2,ymin=-3,ymax=1):

\begin{figure}
\centering
\pandocbounded{\includegraphics[keepaspectratio]{images/Adiyatma_23030630062__EMT4Geometry-135.png}}
\caption{images/Adiyatma\_23030630062\_\_EMT4Geometry-135.png}
\end{figure}

Batasan ke garis PQ

\textgreater plot2d(``abs(x+1)+abs(x-1)'',xmin=-3,xmax=3):

\begin{figure}
\centering
\pandocbounded{\includegraphics[keepaspectratio]{images/Adiyatma_23030630062__EMT4Geometry-136.png}}
\caption{images/Adiyatma\_23030630062\_\_EMT4Geometry-136.png}
\end{figure}

\begin{enumerate}
\def\labelenumi{\arabic{enumi}.}
\setcounter{enumi}{4}
\tightlist
\item
  Gambarlah suatu hiperbola jika diketahui kedua titik fokusnya, misalnya P dan Q. Ingat ellips dengan fokus P dan Q adalah tempat kedudukan titik-titik yang selisih jarak ke P dan ke Q selalu sama (konstan).
\end{enumerate}

\textgreater P={[}-1,-1{]}; Q={[}1,-1{]};

\textgreater function d1(x,y):=sqrt((x-p{[}1{]})\textsuperscript{2+(y-p{[}2{]})}2)

\textgreater Q={[}1,-1{]}; function d2(x,y):=sqrt((x-P{[}1{]})\textsuperscript{2+(y-P{[}2{]})}2)+sqrt((x+Q{[}1{]})\textsuperscript{2+(y+Q{[}2{]})}2)

\textgreater fcontour(``d2'',xmin=-2,xmax=2,ymin=-3,ymax=1,hue=1):

\begin{figure}
\centering
\pandocbounded{\includegraphics[keepaspectratio]{images/Adiyatma_23030630062__EMT4Geometry-137.png}}
\caption{images/Adiyatma\_23030630062\_\_EMT4Geometry-137.png}
\end{figure}

\textgreater plot3d(``d2'',xmin=-2,xmax=2,ymin=-3,ymax=1):

\begin{figure}
\centering
\pandocbounded{\includegraphics[keepaspectratio]{images/Adiyatma_23030630062__EMT4Geometry-138.png}}
\caption{images/Adiyatma\_23030630062\_\_EMT4Geometry-138.png}
\end{figure}

\textgreater plot2d(``abs(x+1)+abs(x-1)'',xmin=-3,xmax=3):

\begin{figure}
\centering
\pandocbounded{\includegraphics[keepaspectratio]{images/Adiyatma_23030630062__EMT4Geometry-139.png}}
\caption{images/Adiyatma\_23030630062\_\_EMT4Geometry-139.png}
\end{figure}

\backmatter
\end{document}
